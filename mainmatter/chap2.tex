\chapter{Background}
\label{chap:background}


\section{Theories of speech act} \label{sec:bg:theory}




\begin{comment}
The generative tradition avoids this problem by proposing a formal feature or an affix for clause types that resides in $C^{0}$ (or its equivalents). For example \textcite{chomsky1995minimalist} (following \textcite{katzpostal1964, baker1970int}) propose that interrogative sentences have a phonetically null affix Q that prompts subject-auxiliary inversion and \twh-movements. As Q is a clausal feature, it can appear in embedded clauses as well.  Additionally, Q is a strong feature that must be checked by PF (phonetic form). Consequently, Q requires an overt head, realized by auxiliary raising over the subject to $C^{0}$, creating subject-auxiliary inversion. If $C^{0}$ is Q, there is a strong [+wh] in [Spec, CP]. In English this strong [+wh] feature must be checked via specifier-head agreement, triggering the \twh-phrases to move to [Spec, CP]. 

Later Q is replaced by features like [+int], but the intuition that interrogative clauses have a strong feature/affix that is phonetically null and changes the force of the sentence is preserved. Similarly, imperatives are anlayzed as having an [+imp] feature in $C^{0}$. Under this approach, declaratives are the default clause type [$-$int, $-$imp]. 

Additionally, many assume that every clause must be associated with a clause type feature (\cite{chomskylasnik1977, cheng1991} among others). Therefore, learners must figure out what morpho-syntactic cues are associated with each clause type features. We will return to how learners could learn the clause type features and their corresponding morpho-syntactic properties. %But as we have discussed in the previous chapter, since these features are phonetically null, a learner faces a clustering problem.  

This knowledge, that the presence of subject-auxiliary inversion indicates a [+int] and not [-int] value of $C^{0}$, has to be learned from the input and likewise for other surface formal properties relevant for clause typing. This is because the relevant properties indicative of the value of $C^{0}$ (and the clause type category of the sentence) differ from language to language. For example, as we will discuss in detail in the next chapter, Mandarin-acquiring infants, unlike English-acquiring infants, need to pick out that [+int] is expressed by the presence of a sentence-final particle, specifically the particle \tit{ma}, and hence should use this surface feature to cluster sentences into clause types.



Many theories assume that clauses must be ``typed,'' namely identified with their clause type category, to trigger certain operations. For example, \textcite{chomskylasnik1977} assumes that COMP must be identified with [$\pm$ wh] for verb selection purposes. Their intuition is that each clause must be identified as a declarative ($-$wh) or an interrogative (+wh). \textcite{cheng1991} formulates this intuition with the Clause Typing Hypothesis:

\begin{quote}
Every clause needs to be typed. In the case of typing a \twh-question, either a \twh-particle in $C^{0}$ is used or else fronting of a \twh-word to the Spec of $C^{0}$ is used, thereby typing a clause through $C^{0}$ by Spec-head agreement. \hfill \textcite[p.9]{cheng1991}

\end{quote}

While the specific way that a clause can be typed has been debated in the literature, people generally agree that identifying the [$\pm$ int] feature on $C_{0}$ is necessary. This means that, learners of grammar must be able to type a clause. %
\end{comment}




\subsection{Speech acts and their link with clause types} \label{sec:bg:theory:speech}
Frege observes that it is possible to express a thought without judging it to be true, and that considering a particular claim and judging it to be true are different things altogether. In asking a polar question like \tit{Is it raining? }one is not judging anything to be true. But in answering \tit{Yes }or asserting the declarative sentence like \tit{It is raining,} one does. Nevertheless, \tit{Is it raining?} and \tit{It is} raining have something in common: they have the same content, or as Frege puts it, they contain the same \tit{thought}. From this observation he derives the influential distinction between \tit{content} and \tit{force}. The following passage from \tit{Der Gedanke} makes this distinction clear:

\begin{quote}
    

An interrogative sentence and an assertoric one contain the same thought; but the assertoric sentence contains something else as well, namely assertion. The interrogative sentence contains something more too, namely a request. Therefore two things must be distinguished in an assertoric sentence: the content, which it has in common with the corresponding propositional question; and assertion. The former is the thought or at least contains the thought. So it is possible to express a thought without laying it down as true. The two things are so closely joined in an assertoric sentence that it is easy to overlook their separability. 

$\,$\hfill Frege 1918: 62, Translated by Peter Geach and R. H. Stoothoff
\end{quote}

That is, the polar question \tit{Is it raining?} and the assertion \tit{It is raining} both contain the same thought, namely \tit{whether it is raining.}  But \tit{Is it raining?} presents this thought with questioning force and \tit{It is raining} presents it with assertoric force. By presenting a thought with assertoric force one expresses that one judges the thought to be true. But there are other ways of presenting the same thought, for example with questioning force. \footnote{However, Frege argues that since the question of truth does not arise for sentences expressing commands, requests, wishes etc., so even though they have sense, their sense is not a \tit{thought}.}

The idea of force is further developed by Austin in his work on speech acts. The core observation is that utterances may have a variety of forces, not just asserting and questioning, but also promising, declaring, warning, marrying, naming and so on. According to Austin, the same sentence can in principle be used with a broad variety of forces:

\begin{quote}
    
We may be quite clear what \tit{Shut the door} means, but not yet at all clear on the further point as to whether as uttered at a certain time it was an order, an entreaty or whatnot. What we need besides the old doctrine about meanings is a new doctrine about all the possible forces of utterances.

\hfill Austin 1956/1979: 251
\end{quote}

This new doctrine is to categorize utterances primarily by what one \tit{does} with them, rather than what they \tit{mean.} His analysis of what one does with an utterance happens on three levels: the locutionary act (the act of speaking), the illocutionary act (the act done through speaking, e.g. informing, warning, requesting), and the perlocutionary act (the by-product of producing the utterance). Let’s walk through an example with these three levels. If we utter a sentence like \tit{The campus is closed tomorrow}, the locutionary act is the speaker utters these very words; the illocutionary act is that the speaker informs the addressee of the news about campus shutdown, and (at least one of) the perlocutionary acts here is that the speaker is changing the addressee’s plans about coming to campus tomorrow. The illocutionary act is of particular interest, as it is the speaker’s intent to inform by producing the utterance. To be more precise, according to Austin, the speaker and hearer are engaging in a conventional procedure of information sharing in which performing a certain locutionary act \tit{is} to perform a certain illocutionary act. The locutionary act has its illocutionary force in virtue of being a part of this conventional procedure. The utterance has informing force by convention.

Austin is in particular interested in developing a taxonomy of speech acts. He focusses his attention on categorizing the \tit{speech act verbs}: the verbs that can be explicitly used to invoke a particular convention by using the word \tit{hereby}. For example, one invokes the procedure of information sharing (i.e. one makes explicit that one is performing the illocutionary act of informing) by saying \tit{I hereby inform you}. Not all verbs with a roughly communicative meaning are speech act verbs. For example, it is odd to say \tit{I hereby surprise you.} So, \tit{surprising} is not an illocutionary force.

Searle (1969) points out that there is no reason to believe that the verbs of English (or any other language, for that matter) exhaust the illocutionary forces. There could be illocutionary forces that do not correspond to a verb. However, Searle agrees with Austin that speech acts, and in particular their illocutionary forces, involve certain social conventions and constitutive rules. Searle moreover follows Grice (1957meaning) in stressing the role of the speaker’s intention to perform a certain act and to be recognized as performing this act. What makes the intention recognizable, according to Searle, is how the locutionary act is in accordance to certain conventional rules:

\begin{quote}
    

In the performance of an illocutionary act, the speaker intends to produce a certain effect by means of getting the hearer to recognize his intention to produce that effect, and furthermore, if he is using words literally, he intends this recognition to be achieved in virtue of the fact that the rules for using the expressions he utters associate the expressions with the production of that effect. \hfill (Searle 1965, 259)
\end{quote}

Searle (1975, et seq.) then gives the following broad categorization of speech acts by what they are used to achieve.

\bex{spact:searle-tax}
\bxl
assertives = speech acts that commit a speaker to the truth of the expressed proposition
\ex directives = speech acts that are to cause the hearer to take a particular action, e.g. requests, commands and advice
\ex commissives = speech acts that commit a speaker to some future action, e.g. promises and oaths
\ex expressives = speech acts that express on the speaker's attitudes and emotions towards the proposition, e.g. congratulations, excuses and thanks
\ex declarations = speech acts that change the reality in accord with the proposition of the declaration, e.g. baptisms, pronouncing someone guilty or pronouncing someone husband and wife
\exl
\eex

Like Frege, Searle also draws a strict distinction between force and content. However, there are some significant differences between Frege’s and Searle’s views on the distinction. For Frege, content is what the sentence radical expresses (e.g. what’s in common between a declarative and a polar interrogative). But for Searle expressing a content is an act and he does not see how “sentences could perform acts” (rather, acts are performed by speakers, p. 257). Thus, speakers perform illocutionary acts and many illocutions involve the expression of a content.\footnote{Searle explicitly mentions that there are illocutionary acts without contents, such as \tit{Ouch} or \tit{Hurrah!}} The content of an illocutionary act is the proposition that the speaker expresses by uttering a particular sentence. Nevertheless, of course, which sentence the speaker utters (and how) influences what the content and the force of the illocutionary act is. According to Searle, sentences have a proposition-indicating element and a function-indicating device also called, in later work, the illocutionary force indicating device (IFID) of a sentence (Searle 1975). For Searle, IFIDs are formal features including, in English, word order, stress, intonational contour, punctuation, the mood of the verb, and certain lexical elements (notably, Austin’s speech act verbs). However, he also notes that in many circumstances the context alone is sufficient to indicate force, so IFIDs do not always need to be explicit (p. 257). He concludes that there are rules for indicating propositions and rules for indicating force, but that these rules can be investigated independently from one another. 

However, if we taxonomize illocutionary acts by the formal features of the utterances with which they are performed, we encounter the phenomenon of indirect speech acts. Indirect speech acts will play a major role later in this dissertation. It is important to note that it only makes sense to speak of \emph{indirect} speech acts once a taxonomy is in place. In the case of Searle’s taxonomy of speech acts by their associated IFIDs, an indirect speech act is an illocutionary act that has the IFIDs of another illocutionary act. For example, \tit{Can you pass me the salt?} uttered at the dinner table is conventionally understood to perform an act of requesting, but it has the formal features of a question. \textcite{searle1975indirect} insists that such utterances still have the force indicated by their formal features and, thus, that Can you pass me the salt? has the illocutionary force of a question. Nevertheless, Searle claims, one can perform a request by performing a question. In this example, the questioning force is the “secondary” illocutionary act and also the ``literal” meaning of the utterance, whereas and the requesting force is “primary” and “nonliteral” (170). This however breaks with the idea that illocutionary forces are indicated by IFIDs, for the utterance contains no IFID indicating requesting. Although, in the absence of explicit IFIDs the context can fill in the gaps, the theory does not provide for the context to change the IFIDs of an utterance. 

\textcite{searlevanderveken1985} then pursue the project of developing a taxonomy for all possible forces (i.e. types of illocutionary acts). They suggest that each possible illocutionary force can be defined as a septuple of values, each of which is a “setting” of a value within one of the seven characteristics of illocutionary acts: the illocutionary point, its degree of strength, the mode of achievement, the content conditions, the preparatory conditions, the sincerity condition, and the degree of strength of the sincerity condition. It follows, according to this suggestion, that two illocutionary forces F1 and F2 are identical just in case they are characterized by the same setting of these seven values.

However, as any group of annotators who attempt to label corpus data using any of the Austinian-Searlean methods for defining and determining speech act types will tell you, such definitions are extremely finicky. Consequently, achieving any sort of inter-annotator agreement is next to an impossible task. This problem already manifests in some way in our previous example about the utterance \tit{the campus is closed}. The formal features of the utterance are insufficient to determine the speech act that is being performed. When talking about the example, it seems natural to say that the illocutionary act is informing, but couldn’t we say just as well that the speaker intends to produce an effect of warning with the sentence? Or an effect of lamenting? Without an explicit speech act marker like \tit{inform}, \tit{warn} or \tit{lament} (e.g. in \tit{I’m warning you}), it’s almost impossible to discern these effects, especially as a corpus annotator. The possibility of indirect speech acts exacerbates this issue further, as \tit{the campus is closed} could also be an indirect answer to a question or, indeed, a request to hand over a key, or extend a homework deadline.\footnote{An unfortunate consequence of the multitude of illocutionary forces that one may attribute to the same utterance is that, in order to boost inter-annotator agreement, annotation schemas sometimes blend formal categories with functional categories. For example in the schemas discussed in \textcite{ninio1994} and \textcite{dialogact}, \twh-interrogative and polar interrogatives are included in the labeling.}

At this point, it appears fruitful to attend not to the fine distinctions of Austinian speech act verbs, nor to the distinctions of Searle and Vanderveken’s metaphysics of action, but rather to how utterances affect the context they are made in (\cite{hamblin1971, stalnaker1978, lewis1979scorekeeping, gazdar1981speech}). As \textcite{chierchia1990textbook} points out, while a sentence like (\ref{spact:chierchia}) can be uttered to perform many different speech acts, such as claiming, guessing, reminding, warning, or threatening, there is a similarity across all these cases. Namely, that a declarative sentence S places the proposition expressed by S “in the common ground and discards any possibilities rendered no longer live because of their inconsistency with that (possibly new) information” (\cite[p.171]{ chierchia1990textbook}). 

\bex{spact:chierchia}
The bull is in the field.
\eex

This way of conceptualizing speech acts not in terms of the rules governing their use, but in terms of their discourse effect is sometimes referred to as the “discourse dynamics tradition” (\cite{murraystarr2020}). Instead of trying to pin down the rules for each speech act, this tradition looks at their conventionalized effects on the discourse. 
In this tradition, the link between clause type and meaning is essential. \textcite{chierchia1990textbook} to refer to this link between grammar and meaning as \tit{sentential force}. As \textcite{chierchia1990textbook} puts it, sentential force is ``what the grammar assigns to the sentence to indicate how that content is conventionally presented'' (p.164).\footnote{Some use the term \emph{sentence mood} (e.g. \cite{portner2018}). But looking at the definition given by \textcite{portner2018}, it seems that these two refer to the same thing:
\begin{quote}
Sentence mood is an aspect of linguistic form conventionally linked to the fundamental conversational functions within semantic/pragmatic theory.\\
\hspace*{\fill} \hfill \textcite[p.122]{portner2018}
\end{quote}
\cite{portner2018} also uses the term \tit{sentential force}, but to refer to the conventional effects that an utterance have on the discourse.  
} 
For example, the conventionalized effect of an assertion on the discourse could be described as proposing to add a proposition to the common ground (CG, \cite{stalnaker1978assertion,stalnaker2002cg}). Similarly, the questioning speech act can be described as adding a question to a stack of questions Question under Discussion (QUD, \cite{roberts1996, ginzburg1995-1}), and requests/commands could be adding an item to the To-do-list (\cite{portner2004}). Table~\ref{tab:portner2004} summarizes the clause types and their canonical functions, and their conventionalized effects on the discourse.

\begin{table}[H]
\begin{center}
\begin{tabular}{l|l|p{8cm}} 
\hline 
& Canonical Function & Conventionalized effect on the discourse \\
\hline
Declarative & Assertion & Proposing to add proposition to the common ground \\ 
\hline
Interrogative & Question & Add current question to the Question Under Discussion stack \\
\hline
Imperative & Request & Add (or propose to add) certain property to the To-Do List of the addressee \\ 
\hline
\end{tabular} 
\end{center}
\caption{Clause types and their conventionalized force; adapted from \textcite[p.238]{portner2004}}
\label{tab:intro:portner2004}
\end{table}



Regardless of theoretical assumptions, what can be agreed on is that languages tend to have three major clause types (\diis{}), and they are linked to three major speech acts (\aqrs{}). In this dissertation, I will follow \textcite{chierchia1990textbook} to refer to this link between grammar and meaning as \tit{sentential force}. As \textcite{chierchia1990textbook} puts it, sentential force is ``what the grammar assigns to the sentence to indicate how that content is conventionally presented'' (p.164).\footnote{Some use the term \emph{sentence mood} (e.g. \cite{portner2018}). But looking at the definition given by \textcite{portner2018}, it seems that these two refer to the same thing:
\begin{quote}
Sentence mood is an aspect of linguistic form conventionally linked to the fundamental conversational functions within semantic/pragmatic theory.\\
\hspace*{\fill} \hfill \textcite[p.122]{portner2018}
\end{quote}
\cite{portner2018} also uses the term \tit{sentential force}, but to refer to the conventional effects that an utterance have on the discourse.  
} As the clause type information can be thought of as a feature on $C^{0}$ (or specifically the head of ForceP following \cite{rizzi1997}), sentential force can be thought of as the semantics of the features [\textpm int, imp]. I will loosely use the term ``speech act'' to refer to the conventionalized effects that the whole utterance has on the discourse, which could be different from the sentential force of the sentence. For example, I consider the indirect speech act example \tit{can you pass the salt} an interrogative clause with [+int] feature, that has interrogative sentential force, but the whole utterance expresses a requesting speech act. 



%People also generally agree that assertions are related to 



\subsection{Complication: the role of prosody}
\label{sec:bg:theory:prosody}

The precise relationship between prosodic features, clause type, and speech act may not be so straightforward. Cross-linguistically, rising contour is usually associated with a questioning act (\citealt{bolinger1978, ladd1981, gussenhovenchen2000, ladd2001typology}, a.o.). In English, declaratives tend to be associated with falling intonation, polar interrogatives tend to bear rising intonation, while \twh-interrogatives also generally bear a falling contour (\citealt{ladd1981, hedberg2014corpus}). When declaratives are associated with rising intonation, they tend to be interpreted as questions (i.e. rising declaratives, \citealt{ladd1981,gunlogson2004,gunlogson2008,jeong2018,rudin2018,goodhue2021rd}). The main point of contention comes down to whether sentences like (\ref{ex:bg:theory:prosody:rd}) are declaratives or not (I use $\nearrow$ to indicate the sentence has a final rising intonation, and $\searrow$ for final fall):
 
\bex{ex:bg:theory:prosody:rd}
It's raining $\nearrow$
\eex
\bex{ex:bg:theory:prosody:fd}
It's raining $\searrow$
\eex


In Gunlogson's \cite*{gunlogson2008} analysis, sentences like (\ref{ex:bg:theory:prosody:rd}) are declarative clauses with a rising intonation, implying that intonation does not affect the clause type feature of the sentence. Meaning-wise, clause type and intonation contribute compositionally to the discourse function of the utterance. 


Many analyses of rising declaratives in the dynamic discourse tradition adopt the same assumption. As these cases are rarely discussed in syntactic literature, it is unclear whether syntacticians would classify them as having [$-$int] or [$+$int]. But we can go through the arguments given to other [$+$int], to see if it's possible to assign [$+$int] to the sentence. One test is whether rogative verbs that only select interrogative clauses would embed (\ref{ex:bg:theory:prosody:rd}). If (\ref{ex:bg:theory:prosody:rd}) bears [+int], then the sentence with (\ref{ex:bg:theory:prosody:rd}) embedded under \tit{wonder} should be grammatical.

\bex{ex:bg:theory:prosody:rd-wonder}
*Mary wonders \tun{it's raining}.
\eex

As shown in (\ref{ex:bg:theory:prosody:rd-wonder}), this prediction is not borne out; (\ref{ex:bg:theory:prosody:rd}) cannot be embedded under rogative verbs like \tit{wonder}. This suggests that (\ref{ex:bg:theory:prosody:rd}) does not have the feature [+int]. But it could be that (\ref{ex:bg:theory:prosody:rd}) has a silent [+int], but this silent [+int] is realized as \tit{whether} in embedded context. So instead of (\ref{ex:bg:theory:prosody:rd-wonder}), embedded (\ref{ex:bg:theory:prosody:rd}) should be:

\bex{ex:bg:theory:prosody:rd-whether}
Mary wonders \tun{whether it's raining}.
\eex

As the rising intonation associated with (\ref{ex:bg:theory:prosody:rd}) would not manifest in embedded clauses, it's hard to distinguish the two possibilities syntactically. But \textcite{farkasroelofsen2017} observe that if the embedded clause is preposed, the intonation difference is preserved, and rising declaratives pattern with polar interrogatives in these cases:

\bex{ex:bg:theory:prosody:rd-prepose}
\bxl{}
`It's raining$\searrow$,' it appears/*she wondered.
\ex`Is it raining?,' *it appears/she wondered.
\ex`It's raining$\nearrow$,' *it appears/she wondered.
\exl
\eex

\textcite{farkasroelofsen2017} uses this evidence to argue that rising declaratives are similar to interrogatives (at least in semantics). They propose that the syntactic representation of sentence forms have two kinds of clause type markers, \tsc{closed/open} and \tsc{dec/int}, as in (\ref{fig:bg:fb2017}). Although they didn't specify where these markers reside in the syntactic structure, but they seem to adopt a split-CP approach following \textcite{rizzi1997}, and thus the two clause markers are both part of CP. 


\begin{figure}[H]
\begin{center}
\begin{tikzpicture}[level distance=30pt]
\tikzset{level 1/.style={sibling distance=15pt}}
\tikzset{level 2/.style={sibling distance=35pt}}
\Tree
[.$$ \tsc{open/closed} [ \tsc{int/dec} [.TP ] ] ]
\end{tikzpicture}
\end{center}
\caption{Extended structure of CP proposed by \textcite{farkasroelofsen2017}}
\label{fig:bg:fb2017}
\end{figure}

While the morpho-syntax of a sentence is linked to \tsc{dec/int}, the intonation (rise vs. fall) signals whether C bears the feature [\tsc{open}] or [\tsc{closed}]. Rising declaratives like (\ref{ex:bg:theory:prosody:rd}) would be an ``open declarative,'' with the following structure:

\begin{figure}[H]
\begin{center}
\begin{tikzpicture}
\tikzset{level 1/.style={sibling distance=35pt}}
\tikzset{level 2/.style={sibling distance=35pt}}
\Tree
[. CP 
	[. \tsc{open} ]
	[. \tsc{}
		[. \tsc{decl} ]
		[. TP \edge[roof]; {it's raining} ] 
	]
]
\end{tikzpicture}
\end{center}
\caption{The structure of rising declaratives, adapted from \textcite{farkasroelofsen2017}}
\label{fig:bg:fb2017rd}
\end{figure}

So for \textcite{farkasroelofsen2017}, the clause type category of rising declaratives is not exactly declarative with [$-$int], and its semantics is even similar to that of the polar interrogatives. But as many have noted, there might be two types of rising declaratives. The one discussed in \textcite{farkasroelofsen2017} is the inquisitive one, but there might an assertive one (\cite{jeong2018, goodhue2021rd}), as shown in (\ref{ex:engcl:annt:rd:a}):

\bex{ex:engcl:annt:rd-cont}
\bxl
\label{ex:engcl:annt:rd:q}
\tit{S and A are on their way to a birthday party for the daughter of A’s friend. They stop at a store to get a birthday card. As they are both scanning the display for a card for the correct age, S is trying to remember how old the girl has just turned, and he thinks he remembers A telling him that she just turned nine, but he wants to confirm it.}\\
\tbf{S: She’s nine$\nearrow$}
\ex
\label{ex:engcl:annt:rd:a}
\tit{S is enrolling his daughter in a summer camp program with the camp organizer A.}\\
S: I want to sign her up for Spanish classes in the mornings, and rock climbing in the afternoons.\\
A: Okay, there are limited places in each activity based on age group, and some of the age groups have already filled up for rock climbing. How old is your daughter?\\
\tbf{S: She’s nine$\nearrow$}
\exl
\hspace*{\fill}\hfill ex. (12-13), \cite[p.955]{goodhue2021rd}
\eex

In (\ref{ex:engcl:annt:rd:q}), the speaker uses the utterance to elicit a confirmation from the addressee regarding the age of the birthday girl, so the main goal is to solicit responses, similar to that of questions. But in (\ref{ex:engcl:annt:rd:a}), the speaker uses the rising declarative to answer the question raised by the addresseee, proposing to add the proposition \tit{she's nine} to the common ground of the conversation. Even though there is an additional effect associated with the utterance (something to the effect of ``Is there still room in the 9-year-old’s rock climbing group?''), this additional effect is not the main goal of the utterance. \footnote{There is another incredulous use of rising declarative, but the two uses mentioned here are more relevant for the current discussion on the speech act of rising declaratives, as most agree that incredulous rising declarative is also ``inquisitive'', same as (\ref{ex:engcl:annt:rd:q}).} Thus, we might not need to assume an interrogative semantics for rising declaratives.\footnote{Note that if we paraphrase (\ref{ex:engcl:annt:rd:a}), we cannot use \tit{wonder} either:

\bex{ex:bg:theory:prosody:rd-prepose2}
\tit{S is looking a flyer looking for Spanish speakers.}\\
``I speak Ladino $\nearrow$," S thinks/$^{??}$ S wonders.
\eex
 
This suggests that the conclusion we can draw from the test with pre-posed embedded clause might need to be re-examined.}  As \textcite{goodhue2021rd} demonstrates, it is possible to give a unified account for these two uses of rising declaratives without assuming a separate layer of clause type markers. Therefore, in this dissertation, I assume that rising declaratives are declaratives with [$-$int] in $C^{0}$. Since nobody has argued that interrogatives with falling intonation should be classified as having a different clause type as interrogatives with rising intonation, I assume that intonation can modify the conventional effects of a clause (and hence the speech act expressed), and maybe even serve as a cue for identifying clause types, but [\textpm int] does not manipulate intonation the same way it manipulates, say, the word order of subject and auxiliary.

%%BUt going beyond English, there are language that use prosody to disnguish clause types...European portugese...French


\section{Acquisition of speech acts and clause types}
\label{sec:bg:acq}
\subsection{Early communicative abilities}
\label{sec:bg:acq:pre}
Previous studies have shown that infants can recognize communicative intentions even in their first months of life, and they can distinguish signals for communication, such as eye contact, intentional pointing, and speech, from non-communicative signals. Infants tend to follow the gaze of their interactive partner. Newborn babies already show rudimentary form of gaze following (\cite{farroni2004gaze}); 6-month-olds follow gazes that are communicative, such as adults' directly gaze at the object, or being greeted (e.g. ``Hello!'') in infant-directed speech (\cite{gredeback2008gaze, senju2008gaze}). Human infants, but not chimpanzees understand pointing as intentional (\cite{pika2006point, povinelli1997point,morissette1995joint}); 9-month-olds remember different aspect about an object depending on whether the object is presented in a communicative (such as pointing) vs non-communicative scenario (\cite{yoon2008intent}); 12-month-olds not only follow the direction of pointing, but also infer that the pointing is to indicate information such as where a toy is hidden in hide-and-seek games (\cite{behne2005hide,behne2012point}). Starting from six months old, infants can readily interpret speech as indicating communicative intentions, as oppose to other vocalizations like coughing (\cite{vouloumanos2014intent}).


Besides recognizing a signal as communicative, infants can attribute goals and desires to an agent, and use shared experience to infer their goals. By 9 months old, infants can recognize an agent's goal of reaching for an object (\cite{woodward1998goal,baldwin2001goal});\footnote{But studies show that still have difficulty with the avoidance type of goals until 14 months old (\cite{feiman2015goals}).} 14-month-olds can distinguish actions that are intentional and ones that are accidental, and only imitate actions that are intentional (\cite{carpenter1998intent,sakkalou2013goal}), and if things go wrong, they offer to help achieve others' goals (\cite{warneken2007goalrecog}); they could use shared experience such as cleaning up toys to interpret an agent's pointing as putting away the pointed object away, and would not attribute this goal to an agent who don't have the shared experience (\cite{liebal2009goal}).

Relatedly, it seems that infants have some understanding of other people's desires and beliefs, attitudes that frequently associated with the speech acts of assertions, questions, and requests. For example, 9-months-old can distinguish whether someone is unwilling or unable to hand them a toy (\cite{behne2005goal}); 12-month-olds can use eye gaze and positive affect to infer which object an agent is going to reach (\cite{phillips2002gaze}), and by 18 months old, infants use positive affect alone to infer which object the agent prefers, even if the preference is different from their own (\cite{repacholi1997desire}). Around 12-month-olds use pointing not to merely direct others' attention, but to request others to share information (\cite{kovacs2014request}); 13-month-olds seek information to clarify ambiguous situations (\cite{vaish2011request}); by 18 months old, infants are shown to be able to attribute beliefs to other people (\cite{onishi2005tom,surian2007tom,song2008earlytom,song2008false, scott2009tom,perner2012earlytom}, see \cite{scott2017review} for a recent review). 18-month-olds could use common ground knowledge shared with an agent to infer whether the agent is requesting an action with an object (e.g. open the door with the key) or simply playing with the object (\cite{schulze2015indirect}).%16-mo use pointing to solicit informaiton (\cite{begus2012point}); 2-year-olds are shown to use what others know when requesting help (\cite{oneill1996knowledge}). 
 


Infants also have some understanding of the norms of conversations. For example, from early on, they understand that conversations take turns. Longitudinal studies with 3 to 5-month-olds show that the overlap between infants' vocalization and parents' speech gradually decrease (\cite{hilbrink2013turn3mo}), and the rhythm of their vocalization starts to mimic the structure of a conversation, as they wait for parents to finish their turn (\cite{hilbrink2013turn,hilbrink2015,casillas2016corpus}); they also use turn transition points (e.g. questions) to cast predicative eye gaze to the next speaker (\cite{casillas2017turn}).

 
 %\textcite{} finds that 
 
In sum, this body of work suggests that by 18 months old, the communicative abilities needed to acquire the distinctions between speech acts and map them to clause types are in place. In the next section, we will see that indeed by this age, infants seem to figured out the major speech acts and are able to link them to their canonical clause types.



\subsection{Early knowledge of clause types and speech acts} \label{sec:bg:acq:spcl}

Previous studies show that children seem to have knowledge about clause types and the associated speech acts (in particular interrogatives and questions) from as early as 18 months old. 

Using corpus data, many studies show that infants can use one- and two-word utterances to express a variety of intentions that can be interpreted as \aqrs{} (\cite{bateson1975,bates1976language,ninio1994} among others); English-speaking children start producing interrogatives around 20 months starting with polar and \twh-interrogatives (\citealt{tyack1977, stromswold1995, rowland2003cdswh}). From the comprehension side, results from corpus studies on parent-child interactions show that children begin to respond to parents’ questions, especially \tit{who, what, where} questions, appropriately around one and a half years old (\citealt{ervintripp1978, steffensen1978, shatz1978comprehension, shatz1978communicative, berningergarvey1981, shatzmccloskey1984, clark2015turn, moradlou2020} among others). 





Results from experimental studies show that around 12 months, English-speaking infants show sensitivity to differences in word order (\citealt{geffenmintz2015wordorder}) associated with declaratives and polar interrogatives respectively. Data from preferential looking tasks show that already at 15 months old, infants look at the objects corresponding to the answers of \twh-interrogatives (\citealt{seidl2003wh, gagliardi2016wh, perkins2020filler}), though their success might not necessarily reflect knowledge of \twh-interrogatives, as they might be relying on their knowledge of verb argument structure in the task (\citealt{perkins2019}). 

By 18 months old, infants seem to be able to distinguish questions from assertions, and use this distinction to infer the epistemic state of the speaker. \textcite{luchkina2018infant} test how well 18-month-olds’ learn labels of novel objects from speakers who made statements about familiar objects (e.g., \tit{This is a star}) and from speakers who asked questions (e.g., \tit{Is this a star?}). They find that even though both types of speakers make statements when labeling novel objects during the test phase, 18-month-olds only learn novel labels from the speaker who made statements during the familiarization phase, but not from the question-askers. This result suggests that 18-month-olds can differentiate questions from statements, and further use this distinction to infer that question-askers might be less reliable information source than statement-makers. 

\textcite{marshmallowqueen} demonstrate that 18-month-olds can associate interrogative syntax with the speech act of questioning with the preferential looking paradigm. In the experiment, infants watch a video of two puppets cheering a mechanical arm for delivering cookies into a box. During the test phase, one of the puppet leaves the scene while the cookie is being delivered, and is thus ignorant of whether there is a cookie in the box. Then participants either hear a polar interrogative \tit{is there a cookie in the box?} or a declarative \tit{there's a cookie in the box!} that could be uttered by either puppet. They find that at 18 months old, infants look more at the ignorant puppet when hearing the polar interrogative, suggesting that they understand the polar interrogative as a question uttered by the ignorant puppet.

\textcite{casillas2017turn} tap into children's knowledge of turn-taking to see if children differentiate questions from other types of speech acts. They measure how often children switch their gaze to track the upcoming speaker. Since questions are turn-transitioning points where a switch in speaker is mostly likely to happen, if children can identify questions, they should switch gazes to the upcoming speaker more often after questions. In the experiment, children are shown videos of a conversation between two puppets in one of four conditions: with normal speech, with words only (filtering out prosodic cues), with prosody, or with no speech. They find that even at 1 years old, children are faster at switching their gaze to the addressee after questions than non-questions in general. Breaking down the different cues to questions, they find that in the word-only condition, even 1-year-olds switch their gaze to the upcoming speaker more often after questions, but in the prosody-only condition, 5-year-olds still are at chance distinguishing questions from non-quesitons. These results suggest that 1-year-olds might be able to use the morpho-syntactic properties of interrogatives to infer questions. Similar results are replicated with 2.5-year-olds by \textcite{lammertink2015turn}.


As for imperatives and commands, \textcite{} show that even in the ``telegraphic speech'' phase (15-30 months old), children respond appropriately to imperatives as commands. \textcite{orfitellihyams2012subj} show that 2.6-year-olds can use the presence/absence of verb morphology to tell whether the sentence is a pro-drop declarative or an imperative. 

Previous studies on questions in speech to children suggest that parents use questions in many different ways (\citealt{holzman1972, shatz1979, tamir1980, yu2019pedagogical}). In particular, compare to questions in adult-directed speech (\citealt{stivers2010}), parents tend to use question as a pedagogical strategy. \citealt{zaitsu2020} investigates the frequencies of the three basic clause types and the corresponding speech acts in speech to children between the age 1 to 3 from the Providence corpus (\citealt{ProvidenceCorpus}) of CHILDES (\citealt{CHILDES}). The results show a relatively robust association of declaratives with assertions, but less so of interrogatives and questions due to questions being often asked via rising declaratives, and requests made with interrogatives. %%still a little odd



Going beyond English, Mandarin-speaking children are observed to produce \ma-interrogatives, A-not-A interrogatives, and \twh-interrogatives around 2 years old (\citealt{miao1986acq, miao1992, lee1989acq, litang1991int, lichen1997compprod, lichen1997comp, fan2012, lijingwong2017}). Experimental studies testing children’s comprehension suggest that Mandarin-children have adult-like interpretation of \twh-interrogatives around 3 (\citealt{fahn2003acq}), but not of polar interrogatives (\citealt{moradlou2020}). 

In our prior work, we find that Mandarin-speaking three-year-olds have adult-like interpretations of \twh-phrases, of both the interrogative and the non-interrogative uses. Our first experiment tested sentences like (\ref{bg-acq:dou}) and (\ref{bg-acq:ba}), where the interrogative and non-interrogative interpretation of the \twh-phrase is disambiguated by morpho-syntactic information (the presence/absence of the particle \dou{}):

	
\bex{bg-acq:dou}
\gll Xiaoyang 	\tun{shenme} 	\tit{dou} 	fangzai 	xiangzili 	le.\\
Lamb	what	\Dou{}	put 	box		\Asp{}\\
``Lamb put everything in the box.''
\eex
\bex{bg-acq:ba}
\gll Xiaoyang	\tit{ba} \tun{shenme} 	fangzai 	xiangzili 	le\\
	Lamb	\tsc{ba}	what	put	box		\Asp{}\\
	``What did Lamb put in the box?''
\eex


Our results show that children, like adults, respond with “yes/no” to sentences like (\ref{bg-acq:dou}) but not to (\ref{bg-acq:ba}), suggesting that they can treat the former as an assertion and the latter as a \twh-question. In a second experiment, we present 3-year-olds and adults with negated sentences like (\ref{bg-acq:negwh}) where the sentence is ambiguous between an assertion and a \twh-question. In this case, the two interpretations are disambiguated by prosodic cues: assigning prosodic prominence on \tit{shenme} gives rise to a \twh-question interpretation, and without prosodic prominence, the sentence is an assertion. Our results again show that 3-year-olds, like adults, can access both interpretations of the sentence.


\bex{bg-acq:negwh}
\gll Xiaoyang mei	fang	\tun{shenme} shuiguo zai	xiangzili\\
Lamb \Neg{}	put	what		fruit		in	box\\
\trans a. ``What fruit didn't Lamb put in the box?''\\
b.	``Lamb didn't put any fruits in the box.''
\eex

\section{Summary}
In this chapter, I reviewed the different theories on clause type and speech act. I also show that regardless of the theoretical stance, people agree on the three major clause types and their corresponding three major speech acts. 

Additionally, results from previous studies show that by by 18 months old, infants seem to have figured out the major clause types and speech acts, and the link in-between. These studies give us a developmental window for our models, namely that our models need to take into account what infants know around 18 months old, to simulate how they figured out clause types and speech acts at this age. In the next chapter, I delve into the problem of how 18-month-olds figure out clause types.


%In summary, children might face many challenges when learning interrogatives and their mapping to questions: English-speaking children should identify word order as a cue for interrogativity, but this information might be buried under many exceptions; Mandarin-speaking children should identify the markers for interrogativity (e.g.~\twh{}) despite some of these markers occurring in non-interrogative environments. Moreover, in both languages, children have to recognize the link between interrogativity and questionhood, which might be masked by exceptions in their input. So, how do children acquire interrogatives and their relationship to questions? In the next section, we will see that despite the noisiness of the input data, there is evidence suggesting that children nevertheless have adult-like understanding of interrogatives, questions, and their relationship at around three years old if not younger. 



 %\cite{evans2014ids} \cite{bernicot1987imp}

