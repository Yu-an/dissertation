% -*- mode: latex; coding: utf-8; fill-column: 72; -*-


\chapter{Introduction}
\label{chap:introduction}
We use language to perform various kinds of speech acts -- providing information, asking questions, making requests, etc. In any language, there are specific signals in the form of a sentence that indicate what speech act it is typically used for. In particular, languages tend to have dedicated types of sentences for the same three basic speech acts: declaratives are typically used for assertions (\ref{ex:intro:intro:dec}), interrogatives for questions (\ref{ex:intro:intro:int}), and imperatives for commands (\ref{ex:intro:intro:imp}):

\bex{ex:intro:intro}
\bxl \label{ex:intro:intro:dec}
That's Elmo. \hfill Declarative, Assertion
\ex\label{ex:intro:intro:int} Is that Elmo? \hfill Interrogative, Question
\ex\label{ex:intro:intro:imp} Find Elmo! \hfill Imperative, Request
\exl
\eex


These dedicated types of sentences are commonly referred to as \tit{clause types} (\citealt{sz1985speechact, konig2007, aikhenvald2016, portner2018}, see \cite{konig2020} for a recent review). Different, unrelated languages tend to have the same clause types, but languages differ greatly in how clause types are realized in the surface form. Even if learners have innate knowledge of what the clause type categories \emph{are}, they still need to learn to \emph{associate} the surface expressions of their language with the right clause type. Previous experimental results suggest that by 18 months old, children differentiate these clause types and associate them with their canonical speech act (e.g. \cite{casillas2017turn,marshmallowqueen}). How do children figure it out? %In principle, the association of clause types with particular speech acts should help here. If learners could perceive speech act types directly, this would indeed be a very useful cue to learn how clause types are expressed in their language. But speech acts cannot be perceived directly and, conversely, we often identity a sentence to be performing a particular speech act by the sentence's clause type. There is a potentially vicious circularity here: we typically identify speech acts by using clause type information, but we might need speech act information to learn to recognize clause types. How do learners avoid this circularity?

In the remainder of this chapter, I will clarify how to understand the notions of clause types and speech acts, and discuss some of the specific problems that learners face when acquiring clause types. I will close by outlining the \emph{pragmatic bootstrapping hypothesis} as a potential solution to the puzzle.

\section{Clause types}
\label{sec:bg:theory:clause}

 In this dissertation I use the term \emph{clause type} to specifically refer to the way of grouping sentences grammatically that are closely associated with the speech act used to perform by the sentence (e.g. assertion, question, request). Clause types are distinguished strictly based on formal criteria, and clause type categories are formal categories. This is different from the traditional view according to which clause type should be thought of as a form-function pair: ``[w]hen there's a regular association of form and the speaker's use of sentences, we will speak of the form-use pair as a sentence type'' (\cite[p.156]{sz1985speechact}). Conceiving clause types as a form-use mapping runs into problems with embedded clauses like the underlined clause in (\ref{ex:bg:theory:cl:embed}), as we cannot tease apart the ``use'' of  this portion of the sentence alone:
\bex{ex:bg:theory:cl:embed}
Mary knows \tun{who Ann can hug}.
\eex

As formal categories, clause type categories participate in syntactic relations like syntactic  selection. Certain verbs select one clause type and not the other. For example, \tit{wonder} selects interrogatives but not declaratives, and \tit{think} selects declaratives but not interrogatives:

\bex{ex:intro:embed:wonder}
\bxl
Mary wonders who Ann can hug.
\ex *Mary wonders Ann can hug Sue.
\exl
\ex \label{ex:intro:embed:know}
\bxl
Mary knows who Ann can hug.
\ex Mary knows Ann can hug Sue.
\exl
\eex

As shown by the contrast between (\ref{ex:intro:embed:wonder}) and (\ref{ex:intro:embed:know}), \tit{wonder} is grammatical with embedded interrogatives, but not declaratives, while \tit{know} is grammatical with both.



Languages tend to have three major clause types (\diis{}) associated with three major speech acts (\aqrs{}), as we have seen in (\ref{ex:intro:intro}). Among these three clause types, declaratives are usually considered the default clause type, whereas interrogatives and imperatives are the results of some operations on declaratives (\cite{sz1985speechact, chomsky1957,chomsky1995minimalist, akmajian1984clausetype, platzack1997imp,rizzi1997} among many others). 



To capture these two language universals, the clause type information is often analyzed as related to an abstract feature occupying $C^{0}$ (\cite{chomsky1995minimalist, cheng1991, rizzi1997, rizzi2001int, chomskylasnik1977,platzack1997imp,akmajian1984clausetype, han1998imp}). An interrogative clause has the [+int] value, imperative [imp], and declarative [$-$int].\footnote{In many analyses, the value for imperative $C^{0}$ is [imp] instead of [$+$imp], partly due to there is no [$+$int, +imp] clause type (\cite{platzack1997imp,han1998imp} among others). In this dissertation, I do not make any commitment on the specific analysis of imperatives.} 




%Transition sentence!!!!
In order to get the clearest picture of the puzzle described above, it is useful to start by taking the strongest position about the clause type categories, i.e. that there are three possible values of $C^{0}$. One can then ask how the learner could identify the signals to these categories. Specifically, the learners need to learn how to \tbf{cluster} sentences into these three clause types, and \tbf{label} them with their functions. 

\section{Learning problems related to clause types}


Assuming that the clause type categories are innate, learners only need to identify the signals to the abstract category. Namely, what learners need to figure out is which sentences have the [+int] feature and which ones have [-int]. 
For example, while (\ref{ex:intro:cluster-base}) and (\ref{ex:intro:cluster:dec}) are two different strings, the learner needs to recognize that they share the same feature when it comes to clause types, frequently analyzed as a [+int] feature on $C^{0}$. They also need to recognize that even though (\ref{ex:intro:cluster:dec}) and (\ref{ex:intro:cluster:int}) share the same lexical items, their $C^{0}$ are marked with different features: [+int] for the former, and [$-$int] for the latter.


\bex{ex:intro:cluster-base}
Do you want a cookie?
\eex
\bex{ex:intro:cluster}
\bxl\label{ex:intro:cluster:int}
Is that Elmo?
\ex\label{ex:intro:cluster:dec}
That’s Elmo!
\exl
\eex


Additionally, learners have to learn the language-specific way that [+int] are expressed. A case in point: in English, the subject and verb switch their canonical order in interrogatives (\ref{ex:intro:intro}a-b), but Mandarin employs sentence final particles for interrogatives (\ref{ex:intro:man}a-b). So learners will have to identify which surface morpho-syntactic properties are relevant from the input.  


\bex{ex:intro:man}
\bxl
\gll Zhe shi Elmo.\\
This is Elmo\\
\trans ``This is Elmo." \hfill Declarative
\ex 
\gll Zhe shi Elmo \tbf{ma}?\\
This is Elmo \Sfp\\
\trans ``Is that Elmo?'' \hfill Interrogative
\ex 
\gll Zhizhi Elmo!\\
Point Elmo\\
\trans ``Point at Elmo!'' \hfill Imperative
\exl
\eex


But [\textpm int] and [imp] of matrix clauses are abstract features and cannot be directly read off from the surface string, so the learner is left with the task of figuring out the surface form of each feature. To achieve this, they need to find the right way to cluster the sentences in their language. In other words, when given a new sentence (\ref{ex:intro:cluster-base}), they should be able to put it in the same clause type category as (\ref{ex:intro:cluster:dec}) and not (\ref{ex:intro:cluster:int}). I will refer to this problem as the \tbf{clustering problem}. 

Solving this problem isn't straightforward, as clause type features of $C^{0}$ do not have a one-to-one mapping with morpho-syntactic properties in the surface form of sentences. For example, as discussed above, the [+int] feature of $C^{0}$ in matrix clauses in English is associated with the raising of auxiliaries, and in many cases, resulting in subject-aux inversion. But in embedded clauses, [+int] does not trigger the raising of auxiliaries, and consequently, we won't see subject-auxiliary inversion in these interrogative clauses: %%should mention subject wh

\bex{ex:intro:eng-embed}
Mary wonders \tun{whether Ann can hug Elmo.}
\eex
\bex{ex:intro:eng-can}
\bxl
Ann can hug Elmo.
\ex Can Ann hug Elmo?
\exl
\eex
As shown in (\ref{ex:intro:eng-embed}), the auxiliary \tit{can} and subject \tit{Ann} of the embedded interrogative have the same word order as the matrix declarative sentence (\ref{ex:intro:eng-can}a).


Conversely, some morpho-syntactic properties typically associated with [+int] could also appear in other settings. For example, [+int] in Mandarin can be used to perform a question by having a \twh-phrase in the sentence. But, [$-$int] sentences could also have \twh-phrases, where these phrases are interpreted as indefinites like English \tit{any/a}. As a result, a string like (\ref{ex:intro:m-whamb}) could be either an interrogative (interpretation a) or a declarative (interpretation b). 


\bex{ex:intro:m-whamb}
\gll Xiaoxiao mei 	chi 	\tun{shenme} dongxi\\ 
Xiaoxiao \Neg{} 	eat	what	things\\
a.	``What didn’t Xiaoxiao eat?''	\hfill Interrogative \twh\\
b.	``Xiaoxiao didn’t eat anything.''		\hfill Indefinite \twh
\eex

Therefore, learners need to infer the abstract clause type feature of sentences, but they might not see the surface morpho-syntactic properties associated with clause type feature, or the property that they do see misaligns with the actual clause type feature of the sentence.

Besides the many-to-many mapping problem, learners also have to deal with cases where the relevant morpho-syntactic cues are masked by operations such as ellipsis.  For example, left-edge-ellipsis could obscure morpho-syntactic cues distinguishing declaratives from interrogatives in English (\cite{zwickypullum1983leftedge}):

\bex{ex:intro:lee}
Want to go out?
\eex

The sentence in (\ref{ex:intro:lee}) could be an ellipsis of  the subject pronoun from a declarative like (\ref{ex:intro:lee-unpack}a), or of the subject and the auxiliary from a polar interrogative like (\ref{ex:intro:lee-unpack}b). But the surface form of (\ref{ex:intro:lee}) itself does not have enough information to help us identify its clause type feature. 

\bex{ex:intro:lee-unpack}
\bxl
You want to go out.
\ex Do you want to go out?
\exl
\eex


%

After identifying the clusters, learners need to determine the canonical function of each cluster in the system. That is, after clustering sentences into three categories, children still need to learn which one of these clusters is the interrogatives, which is the declaratives, and which is the imperatives. We will refer to this as the \tbf{labeling problem}. To break into this labeling problem, the learner might need to use the function of the utterances in a cluster -- that is, the speech act information. As many have noticed, clause types are systematically related to the speech act of a sentence (\cite{katzpostal1964, sz1985speechact, portner2018} among many others). If the learner observes that a cluster of sentences is more frequently associated with questions, they might be able to label this cluster of sentences as the one whose canonical function is asking questions (i.e.\ interrogatives), and similarly for declaratives and imperatives. 

But the mapping between speech acts and clause types is not one-to-one either. The function of a sentence can be thought of as the speech act performed by uttering it. For example, Alex's mother uses (\ref{}) to ask Alex's preferences, performing the speech act of questioning.
\bex{}
Do you want to get down? \hfill Alex's mother, Session 01;05;12, Providence Corpus (\cite{ProvidenceCorpus})
\eex

%This paragraph is not good
Speech act theory focuses on what speakers \emph{do} with sentences (\cite{austin1975things, searle1969} a.o.), but this is not the only way to conceptualize speech acts. I will return to the different options for describing speech acts in the next chapter. For now it suffices to observe that, regardless of one's theory of speech act, it is generally agreed upon that the form of a sentence, i.e. its clause type, constrains the types of speech act performed by uttering it. Canonically, declaratives tend to be assertions, interrogatives questions, and imperatives requests. 


But this mapping from clause type to speech act type is not inviolable. In some contexts, it is possible that the conventionalized speech act associated with a clause type is not the actual speech act performed by uttering it. \emph{Indirect speech acts} are these mismatching cases where the primary, ``non-literal'' force of an utterance is different from the conventionalized, ``literal'' force of a sentence associated with its clause type (\citealt{searle1975tax, searle1976class, bachharnish1979, levinson1983, searlevanderveken1985, portner2018, starr2014, murraystarr2020} a.o.). The common example illustrating this phenomenon is \tit{Can you pass the salt?} as a request. As an interrogative sentence, its conventionalized (and ``literal'') force is questioning, but the primary act performed is requesting. As a result, some speech act categories can be expressed by more than one clause types, and vice versa. For example, interrogatives can express assertions, questions, requests/commands (\ref{eng-cl:q-all}); and questions can be expressed by \diis{} (\ref{eng-cl:int-all}).
\bex{eng-cl:int-all}
Speech acts expressed by interrogatives 
\bxl Is it snowing? \hfill Question
\ex Aren't you sweet. \hfill Assertion
\ex Can you pass the salt? \hfill Request
\exl
\eex

\bex{eng-cl:q-all}
Clause types expressing questions
\bxl
Is it snowing? \hfill Interrogative
\ex It's snowing? \hfill Declarative
\ex Tell me if it's snowing! \hfill Imperative
\exl
\eex


\section{Pragmatic bootstrapping hypothesis}
%It is possible that children could use the link between speech act and clause type to make bootstrap into clause type categories

As said in the initial description of the problem, it would be useful for learners to have access to speech act information in order to determine clause type. After separating the problem into the clustering problem and the learning problem, it may even appear \emph{necessary} that they can access speech act information. Surface features alone may allow a learner to cluster sentences into three distinct formal categories, but without being able to determine the conventional function of the sentences in a cluster they will not be able to \emph{label} the clusters correctly. This leads me to postulate the following pragmatic bootstrapping hypothesis:

\begin{quote}
Infants use the speech act information, in addition to observations of morpho-syntactic features in the surface form of sentences, to cluster and label sentences into the three major clause types.
\end{quote}


%previously...here are some of the differences

But, as said, there is a danger of circularity.
Our \hypos{} assumes that children can infer the speech act categories of the sentence at this point. While some evidence suggests that 18-month-olds can infer speech acts, their inferences might not be perfect. Consequently, they might have only limited access to speech act information. Moreover, there's the problem of how children can infer speech act categories in the first place -- and it is undeniable that the clause type information is useful for solving this problem. As adults, the primary way we identify the act performed by a given sentence is through its clause type. But this is precisely the problem that the child is trying to solve (i.e. identifying the clause type). If children need speech act information to identify clause type categories, but they also need clause type information to identify speech act categories, it seems that we have a chicken-and-egg problem. 

We can break out of the circularity by observing that, to get the learning process going, children do not need \emph{perfect} speech act information. I will evaluate the \hypos{} by 
%I address the first problem by
simulating the learning of clause type with various degrees of noise in the speech act information, so that we can see how much pragmatics a learner needs to succeed at the clustering problem. I then tackle the question of where even imperfect speech act information could come from, if no clause type information is accessible,
%address the second problem
by exploring non-clause type cues for speech act information in the input. I find that the prosody of parents' utterances, speech gap between utterances, and direct eye gaze might be useful in learning the distinctions between speech acts.
These are noisy cues for speech act information, but cues nonetheless. As my simulations show that even very noisy speech act information is useful for learning clause types, this suggests a way out of the circularity. Learners can make use of non-clause type cues towards speech act, which may be sufficient to solve the clustering and labelling problems.

\section{Discussion and roadmap}
\label{sec:intro:roadmap} 

This dissertation is organized as follows. Chapter~\ref{chap:background} examines the developmental trajectory of speech acts and clause types, especially questions and interrogatives. As we will see, English-acquiring infants as early as 18 months seem to have already sensitive to the distinctions between different clause types and speech acts, and seem to understand the mapping between questions and interrogatives. The same holds for infants acquiring other languages as well, even though we have less evidence. Our question then is, how do 18-month-olds learn to figure out clause types?

Chapter~\ref{chap:eng-cl} looks at how English-acquiring 18-month-olds could solve the problem. Specifically, is information from syntax enough for children to find the right three clause type categories, or do they need pragmatic information like the speech act of the sentence to find the right clustering? I build two computational models to address this question, a \distlearner{} (\dlearnerabbr{}), and a \praglearner{} (\plearnerabbr{}). These two learners both need to infer the abstract clause type, but \dlearnerabbr{} draws inferences from syntactic information alone while \plearnerabbr{} uses both syntactic and pragmatic information. I use a corpus study to first provide a quantitative description of the type of input that infants get, and use the resulted annotated dataset as input for the computational models. I find that pragmatic information is indeed important for solving the clustering problem: without the speech act information, \dlearnerabbr{} cannot find the right clause types. Additionally, a little pragmatics goes a long way, as  even if 80\% of the pragmatic information is noise, it still improves the learner's performance. 

In Chapter~\ref{chap:man-cl}, I apply the same methodology to another language, Mandarin. Mandarin-acquiring infants figure out the clause types of their language around the same age as English-acquiring infants, but the two languages employ different morpho-syntactic features for clause typing. How do Mandarin-acquiring infants solve the problem? Do they also need pragmatic information? I compare the same two learners, and found that pragmatics information is crucial for identifying Mandarin clause types; without pragmatic information, learners might not be able to identify any of the clause types correctly.

But in both chapters, I operate under the assumption that infants have information about speech act types at their disposal. How do they obtain such information about the speech acts of their parents' utterances? %Of course their knowledge of clause type might help, but are there signals from other sources?
In Chapter~\ref{chap:eng-sp}, I explore potential cues from prosody and parents' behavior that might help infants identify questions. Chapter~\ref{chap:discussion} concludes the dissertation.

