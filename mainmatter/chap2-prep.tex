As we will see in detail in the next chapter, infants as young as 18 months old seem to have figured it out. Given this early success, However, figuring out what the relevant clause types and speech acts are is not trivial. 


Clause types are grammatical categories that are thought of as features on $C^{0}$ (for example, as a [\textpm int] feature,  cf. \cite{katzpostal1964, baker1970int, langacker1974q, chomsky1995, rizzi1997, rizzi2001int, chomskylasnik1977, cheng1991}), what children need to figure out is which sentences have the [+int] value and which ones take the [-int] value. In English, the [+int] value of $C^{0}$ is associated with the syntactic rule of subject-aux inversion. So the child must come to treat (\ref{engcl:cluster}) and (\ref{engcl:cluster}a) alike (because of subject-aux inversion triggered by [+int] in $C^{0}$) and distinct from (\ref{engcl:cluster}b)

First, there is the clustering problem, namely learners need to be able to identify the abstract clause type category of a sentence. For example, while (\ref{ex:intro:cluster-base}) and (\ref{ex:intro:cluster}a) are two different strings, the learner needs to recognize that they share the same feature when it comes to clause types, frequently analyzed as a [+int] feature on $C^{0}$. They also need to recognize that even though (\ref{ex:intro:cluster}a) and (\ref{ex:intro:cluster}b) share the same lexical items, their $C^{0}$ are marked with different features: [+int] for the former, and [$-$int] for the latter.

%%%%%speech act stuff
- Children need to figure out the conventional effects that certain speech acts have on the current conversation
-- for example, questions require answers, assertions update your belief, requests/commands may add things to your to-do list.
- associate speech act categories with clause types
--- Children need to figure out what speech act is expressed by what kind of clauses with what kind of prosody

The canonical use of questions is to seek information (\citealt{searle1975tax}, \citealt{levinson1983}, see \citealt{krifka2011q} for a recent overview), but when talking to children, especially pre-linguistic infants, parents can’t genuinely expect them to be in a position to provide that information (\citealt{holzman1972, shatz1978comprehension, tamir1980, yu2019pedagogical}, a.o.).


%As our example above shows, the same sentence can be used to perform various speech acts depends on the speaker's intentions and the context under which the sentence is used. While we cannot take the speaker and the context out of the performance of a speech act, it is undeniable that the clause type of a sentence constrains the kinds of speech acts that can be performed.  


One way to think of this connection is to look at how utterances affect the context they are made in (\cite{hamblin1971, stalnaker1978, lewis1979scorekeeping, gazdar1981speech, roberts1996, portner2004, farkasbruce2010} a.o.). 

As \textcite{chierchia1990textbook} points out, while a sentence like (\ref{ex:intro:chierchia}) can be uttered to claim, guess, remind, warn, or threaten, there is a similarity across all these cases. Namely, that a declarative sentence S places the proposition expressed by S ``in the common ground and discards any possibilities rendered no longer live because of their inconsistency with that (possibly new) information'' (\cite[p.171]{chierchia1990textbook}). 

\begin{quote}
Although grammatical entities do not have semantic definitions in adult grammars, it is possible that such entities refer to identifiable semantic classes in parent-child discourse [\ldots] If the child tentatively assumes these syntax-semantics correspondences to hold, and if they do hold, he or she can make the correct inferences in the example. \textcite[p.39]{pinker1984}
\end{quote}

\bex{ex:intro:chierchia}
The bull is in the field.
\eex

Table~\ref{tab:portner2004} summarizes the clause types and their canonical functions, and their conventionalized effects on the discourse.

\begin{table}[H]
\begin{center}
\begin{tabular}{l|l|p{8cm}} 
\hline 
& Canonical Function & Conventionalized effect on the discourse \\
\hline
Declarative & Assertion & Proposing to add proposition to the common ground \\ 
\hline
Interrogative & Question & Add current question to the Question Under Discussion stack \\
\hline
Imperative & Request & Add (or propose to add) certain property to the To-Do List of the addressee \\ 
\hline
\end{tabular} 
\end{center}
\caption{Clause types and their conventionalized force; adapted from \textcite[p.238]{portner2004}}
\label{tab:intro:portner2004}
\end{table}%


Languages typically employ a wide range of clausal constructions -- actives, passives, imperatives, polar and \twh-interrogatives, clefts, pseudoclefts, relative clauses, and so on. Nonetheless, many of these distinctions are collapsed from the perspective of clause typing. For instance, verbs that select for declarative complements do not distinguish between actives, passives and clefts. Children, then, need to cluster the right morphosyntactic properties that are relevant for clause types and associate them with the right pragmatic function.




%%%%%%%%%%%%%%%%%%%%%%%%%
In the principle and parameter approach, a way to solve this cross-linguistic variation problem is to propose a parameter that covers all the variations. The learners are given choices for how to type a clause (i.e. assign [+int] or [-int] to $C^{0}$) by setting the value of a parameter. For example, \textcite{cheng1991} proposes the Clause Typing Hypothesis:

\begin{quote}
Every clause needs to be typed. In the case of typing a \twh-question, either a \twh-particle in $C^{0}$ is used or else fronting of a \twh-word to the Spec of $C^{0}$ is used, thereby typing a clause through $C^{0}$ by Spec-head agreement. \hfill \textcite[p.29]{cheng1991}

\end{quote}


According to this hypothesis, \twh-movement languages such as English assign a [$+$int] ([+wh] in Cheng's terminology, but it includes all interrogative clauses) value to the $C^{0}$ of sentences like (\ref{ex:intro:cheng-eng}) by moving the \twh-phrase to clause-initial position.\footnote{Note in other theories of [+int] and \twh-movement (e.g. \cite{chomsky1995}), \twh-movement does not assign [+int] to $C^{0}$, but is motivated by [+int]. } Meanwhile, \twh-in-situ languages achieve the same goal by utilizing a \twh-particle, \tit{ne}, as illustrated in (\ref{ex:intro:cheng-man}).

\bex{ex:intro:cheng-eng}
Who can Ann hug?
\eex
\bex{ex:intro:cheng-man}
\gll
Ann neng baobao shui (ne)\\
Ann can hug who \Sfp\\
\trans ``Who can Ann hug?''
\eex

%Putting aside the controversy about \twh-particles (cf. \cite{bruening2007wh, yangyang2018}), 
As \twh-particles are optional in many in situ languages, Cheng associates a particular learning strategy with this hypothesis: learners use the presence of polar question particle (e.g. \tit{ma} in Mandarin) to learn

since languages have two ways of realizing [+int], learners identify the abstract clause type feature by observing whether their language allows \twh-particles. 


In this dissertation, we explore a different way of solving the clustering problem, namely whether they can use their ability to track the distribution of certain features. Specifically, learners need to use pragmatic information (i.e. the speech act of the sentence), in conjunction with observations of distributions of morpho-syntactic features in the surface form of sentences, to infer clause type clusterings. We will return to this hypothesis later. 



distinguish interrogatives and declaratives formally by as early as 12 months (\cite{geffenmintz2015wordorder}), and associate interrogatives with questions by age 3, and possibly as early as 18 months (\citealt{tyack1977, ervintripp1978, berningergarvey1981, rowland2003cdswh, seidl2003wh, casillas2013,casillas2017turn, clark2015turn, lammertink2015turn, gagliardi2016wh, perkins2020filler}, among many others). 

Although there have been studies on parents' use of speech acts like questions, how children could use to acquire the distinction between declaratives and interrogatives and their association with their conventional functions remains under-explored. 
 
 By providing data on how parents use questions in child-directed speech, and modeling how children use different signals in this input to learn questions and interrogatives, this project will deepen our understanding of the role of questions in language acquisition and beyond. 

By systematically examining different aspects of language, from syntax, prosody to pragmatics, and modeling the learning process computationally, this study will provide insights into how children come to learn speech acts and clause types. Additionally, by comparing two languages in which the formal features of interrogativity differ greatly, this research will also advance our understanding of the universal and language-specific problems that children face when learning interrogatives. Finally, the labelled datasets as well as the the time-aligned corpora will be made available to the public for research on children's input beyond the learning of speech acts and clause types. 


%prosody
The form of polar questions and the function of Wh-elements proved to be fairly diverse in the input, which required further elaboration of the annotation schema. Polar questions are those which ask whether p or its negation holds. Although rising declaratives (RD) have a similar effect in terms of introducing alternatives p and its negation, they are said to introduce an additional bias for one of the alternatives (Gunlogson 2008, Malamud & Stephenson 2015, Farkas & Roelofsen 2017). Whether they are fundamentally declarative in mood or interrogative has recently been taken up (Farkas & Roelofsen 2017, Malamud & Stephenson 2015, Jeong 2018). We do not take a theoretical stance here, but we did mark them with DECLARATIVE clause type and additionally marked them for rising intonation – they were also given a speech act, which usually (but not always) turned out to be QUESTION, and thus were counted as a MISMATCH. However, under certain approaches (e.g., Farkas & Roelofsen 2017), characterizing a RD as a MISMATCH wouldn’t be wholly accurate, since rising declaratives are viewed as having interrogative mood.
%In English, declaratives tend to be associated with falling intonation, polar interrogatives tend to bear rising intonation, while \twh-interrogatives also generally bear a falling contour (\citealt{ladd1981, hedberg2014corpus}). When declaratives are associated with rising intonation, they tend to be interpreted as questions (i.e. rising declaratives, \citealt{ladd1981,gunlogson2004,gunlogson2008,jeong2018,rudin2018}).  



%In Mandarin, experimental studies controlled for lexical tone find that compared to assertions, in questions, the F0 of the entire utterance is raised (\citealt{ho1977intonation, shen1991question, yuan2002prosody, nikawai2004, lee2005prosody, liuxu2005, yuan2006qprosody, yuan2011perception, liufang2009prosody, jiangchen2011} among many others), but different types of interrogatives might end with different contours: \ma-interrogatives and rising declaratives tend to have a final rise contour (\citealt{shen1991question, zeng2004tones}), and A-not-A and \twh-interrogatives tend to end on falling intonation (\citealt{shen1991question, nikawai2004}, cf. \citealt{liuxu2005,yang2020wh}). Perception experiments show that speakers tend to use the overall F0 of the whole sentence to distinguish questions from assertions (\citealt{jiangchen2011, gryllia2020typing}), and it is more difficult to use clause-final prosodic cues, especially if the final syllable has a rising tone (\citealt{yuan2011perception}). 

To summarize, in both languages, prosodic information alone may not be sufficient to predict the clause type of a sentence. But prosodic information may help explain some of the mismatches between clause type and speech act, as declaratives that function as questions might be associated with rising intonation.



%acq
Studies of children’s early linguistic productions suggest that infants convey a variety of speech acts via one- and two-word utterances (Bates et al. 1975, Camaioni et al. 1998, Ninio 1992, Snow et al. 1996, Chouinard 2007; for an overview, see Cameron-Faulkner 2014). But due to how short these utterances are, such studies don’t offer much insight into children’s understanding of clause type. Perception studies show that infants are sensitive to the formal distinctions between declaratives and interrogatives by 12 months (Geffen & Mintz 2015, 2017, Frota et al. 2014, Soderstrom et al. 2011), however this work does not speak to the interpretations infants assign to these clause types. Shatz (1979) demonstrates that 18-34 month-olds can appropriately reply to around 50\% of questions. In more recent work (Lammertink et al. 2015, Casillas & Frank 2017), children age 2 to 3 are shown to expect conversational turn changes more after interrogatives than declaratives, while 1- to 2-year-old children are only marginally above chance at doing so. This result suggests that interrogatives might be understood as questions, however it is also consistent with a superficial link between the interrogative clause type and change of speaker. Perkins (2019) and others show that 13- and 15-month-olds presented with WH-interrogatives look at the corresponding answer, though this may be due to the argument structure of the verbs instead of the syntax of the interrogative clause type. 18- to 20-month-olds, on the other hand, have a representation of the syntax of WH-interrogatives that is more adult-like (Seidl et al. 2003, Gagliardi et al. 2016, Perkins 2019, Perkins & Lidz 2020, in prep.).

By the time children are 3 years old, there is more robust evidence that they have acquired the associations between the three main clause types and their standard speech acts. In an experimental setting, Rakoczy & Tomasello (2009) demonstrate that by age three, children know that declaratives convey assertions, while by age two they know that imperatives convey requests (on requests, see also Shatz 1978, Liszkowski 2005, Grosse et al. 2010). As for the interrogative-question link, many studies both experimental and naturalistic show that children respond appropriately to interrogatives by age three, suggesting that interrogative sentences are interpreted as questions. Two- to five-year-olds frequently ask “why” and “how” questions (Frazier et al. 2009, 2016). Three-year-olds differentiate test questions from information-seeking questions (Grosse & Tomassello 2012). Prior work conducted by the sponsors of this application argues that comprehension of the link between clause types of embedded clauses and their usual speech act function plays a crucial role in the acquisition of the meaning of attitude verbs like “think”, “want”, “know”, and “hope” around age three (Hacquard & Lidz 2019, Lewis et al. 2017, White et al. 2018, Harrigan et al. 2019, Dudley et al. 2017). Taken together, this research shows that children have acquired speech acts and their clause type mappings by about age three, but leaves open when this first happens.

%marshmallow queen
Participants watch a video of two puppets (window-puppet and free-puppet, see fig. 1). In the training phase, cookies are delivered into the box via a mechanical arm (4 trials). Free-puppet eats the cookie, and both puppets say “Yay!” in unison (one voice male, one female, counterbalanced). Sometimes the arm appears without leaving the cookie behind, and the puppets say “Awww” (3 trials). At pre-test, a phone rings and one of the puppets (counterbalanced) exits the frame before the cookie is delivered. The remaining puppet watches the delivery, then the first returns. Thus one puppet is ignorant of the delivery and so poised to ask a question, while the other knows about the delivery, so is poised to assert. At test, the puppets face each other, and participants hear either a series of questions (“Is there a cookie in the box? Is there a cookie? Is there?”) or a series of assertions (“There’s a cookie in the box! There’s a cookie! There is!”). The puppets don’t have moveable mouths, so the only way to know who is speaking is to combine contextual information with utterance form.

18-month-olds look more at the ignorant puppet during the first series of utterances in the question-first condition than in the as- sertion first condition. These results suggest that 18-month-olds understand interrogatives as questions and declaratives as assertions. In future work, we will test 12-month-olds in search of a lower bound.


%chap 3
Another unexpected cue is post-verbal UFIs has a very correlation with [-int]. Looking at Figure~\ref{fig:real-syncluster}, it seems that the few cases of post-verbal UFIs appear in declaratives rather than interrogatives. These UFIs are not complementizers but \twh{s}:
\bex{ex:engcl:corpus:formal}
You know \tbf{how} to get it out.
\eex

This tendency might be a result of our annotation method, as we only annotated the main clause, due to the assumption that at 18 months, infants might not perceive the clause type of embedded clauses yet. Thus, these sentences were annotated as a matrix declarative with a post-verbal UFI, \tit{wh}. 
%%%%
To determine which cues are informative of the clause type features (i.e whether the $C^{0}$ of a clause has + or $-$ int/imp]), I ran a multinomial logistic regression models with the three clause type as the dependent variable (with declarative as the default value), and the morpho-syntactic cues as independent variable. 


\footnote{Due to extreme low frequency, models with [+pre-verbal UFI] would not converge; since this feature should not predict the use of any clause types, it was excluded in the final model.} 



Before we dive into this tradition, we need to have a quick note on the terminology, as sometimes the same term refers to different notions, both within the discourse dynamics tradition and compare to the classical speech act theories. 

\textcite{portner2018} is the only one explicitly defines what he means by clause type, sentence mood, and sentential force: 
\bex{portner:concepts}
\bxl
\tbf{Clause types} are grammatically defined classes of sentences which correspond closely with sentence moods. (p.122)
\ex\tbf{Sentence mood} is an aspect of linguistic form conventionally linked to the fundamental conversational functions within semantic/pragmatic theory; (p.122) sentence mood is fundamentally a matter of which syntactic features a clause has (p.141). 
\ex \tbf{Sentential forces} are the fundamental conversational functions with which sentence moods are associated. (p.124)
\exl
\eex


We will follow the tradition dates back to Frege (\cite{frege1918thought}) to refer to this link between grammar and meaning as \tit{sentential force} (\cite{chierchia1990textbook}). As \textcite{chierchia1990textbook} puts it, sentential force is ``what the grammar assigns to the sentence to indicate how that content is conventionally presented'' (p.164).\footnote{Some use the term \emph{sentence mood}. But looking at the definition given by \textcite{portner2018}, it seems that these two refer to the same thing:
\begin{quote}
Sentence mood is an aspect of linguistic form conventionally linked to the fundamental conversational functions within semantic/pragmatic theory.\\
\hspace*{\fill} \hfill \textcite[p.122]{portner2018}
\end{quote}
\cite{portner2018} also uses the term \tit{sentential force}, but to refer to the conventional effects that an utterance have on the discourse, what some would call \tit{utterance force}. 
} As the clause type information can be thought of as a feature on $C^{0}$ (or specifically the head of ForceP following \cite{rizzi1997}), sentential force can be thought of as the semantics of the features [\textpm int, \textpm imp].


