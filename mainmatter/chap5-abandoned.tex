\subsubsection{Utterance context}

Learners of both (perhaps all) languages face a further difficulty. Besides their main function of eliciting information, interrogative clauses can also be used to fulfill a range of other conventional functions. Test questions like (\ref{bg-prag:test}) do not present the asker as lacking information; parents can use questions like (\ref{bg-prag:ped}) to teach children object labels; indirect requests like (\ref{bg-prag:req}) are not genuinely soliciting information; and rhetorical questions act similar to assertions (\ref{bg-prag:rhe}). Such uses of interrogatives might mask the association of interrogatives with information-seeking speech acts. Additionally, other clause types like declaratives can be used as questions (e.g. rising declaratives, \citealt{gunlogson2004,gunlogson2008,jeong2018,rudin2018}). All these are potential sources of noise in the input with respect to the mapping between form and function.

\bex{bg-prag:types}
\bxl\label{bg-prag:test}
What is H$_{2}$O?		\hfill \tit{Test}
\ex \label{bg-prag:ped}
What's this?	\hfill \tit{Pedagogy}
\exl
\eex

\bex{}
\bxl\label{bg-prag:req}
Can you pass the salt?			\hfill \tit{Request}
\ex \label{bg-prag:rhe}
Are you crazy?	\hfill \tit{Rhetorical}
\exl
\eex


Common to all the functions listed above is the fact that in a conversation, questions expect responses (\citealt{duncan1972turn}). In turn-taking theory, questions usually mark the turn-transition points: after a question, the current speaker yields their turn and appoint the next speaker while the other participants of the conversation would pick up the turn by answering the question (\citealt{kendon1967gaze, argyle1972gaze, levinson1983, tice2011turn} among others, see \citealt{enfield2010} for a survey).  Additionally, questions are argued to set up the topics and issues in a conversation (\citealt{roberts2012,farkasbruce2010}), and speakers could use questions to keep track of where they are in a conversation. Therefore, despite the fact that questions can be used to fulfill many functions, the role of questions in a conversation seems to be clear: speakers use questions to solicit responses and set up topics for discussion. If children are sensitive to signals of response expectations or topic settings, they could make use of that information to identify questions, and, in turn, interrogatives. 
%As mentioned above, combining syntactic and prosodic information may explain some mismatches, as with the case of rising declaratives. Other syntactic markers might be helpful as well: many have noted that interrogatives that function like imperatives tend to take specific forms such as having modals like (\ref{bg-prag:req}), or \tit{why don't} and \tit{why not} interrogatives (i.e. \tit{whimperatives}, \citealt{sadock1974, green1975whimp}).


we will annotate the social function of each utterance. Our preliminary results in Fig.~\ref{fig:uttgoals} show that this prediction is borne out: parents tend to use questions to direct infants’ attention to new objects in their surrounding, while assertions are used to teach and express opinions.

\begin{table}[H]
\begin{center}
\begin{tabular}{c|p{6cm}|c}
\hline
Social Function&Explanation	&\tit{Example}\\
\hline
\hline
Attention& Direct attention to new object&\tit{Alex, Look!}\\
\hline
Negotiating & Negotiating about carrying out an action	&\tit{You read it to mommy.}\\
\hline
Teaching& Teaching the child about something or how to do things &\tit{What’s that (pointing to a bumblebee)?}\\
\hline
Discussing & Exchanging information but not for pedagogical purposes &\tit{Do you like scratchy cat kisses?}\\
\hline
Verbal Routines& Routines in social situations/games &\tit{Ready? Go!}\\
\hline
Emoting& Expressing emotions like excitement &\tit{Yay!}\\
\hline
Imitating & Imitating sounds, repeating others' utterances &\tit{vroom vroom!}\\
\hline
Meta-communication & Seeking clarification, confirmation, acknowledgement of another utterance/action	&\tit{ (after Alex makes some noise) What?}\\
\hline
\end{tabular}
\end{center}
\label{code:social}
\caption{ Types of social functions of parents’ utterances}
\end{table}

\subsubsection{Overview}
\label{sec:engsp:results:uttgoals}

