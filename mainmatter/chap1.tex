% -*- mode: latex; coding: utf-8; fill-column: 72; -*-
\chapter{Introduction}
\label{chap:introduction}

We use language to perform various kinds of speech acts -- providing information, asking questions, making requests, etc. In any language, there are specific signals in the form of a sentence that indicates what speech act it is typically used for. In particular, cross-linguistically, languages tend to have dedicated clause types for the same three basic speech acts: declaratives are typically used for assertions, interrogatives for questions, and imperatives for commands (\citealt{sz1985speechact, konig2007, aikhenvald2016, portner2018}, a.o.). However, the surface formal features of these clause types differ greatly from one language to the next, and children need to figure out the language-specific surface formal features of their clause types. In this dissertation, I ask how children learn to identify the speech acts of the people speaking to them while learning the surface signals associated with the clause types expressing these acts. 


Learning to identify clause types and speech acts is extremely important for early language acquisition. The acquisition of various basic syntactic phenomena like argument structure, word meanings, and basic word order, might be aided by an ability to distinguish declarative clauses from other clause types (\citealt{pinker1984, pinker1989, gleitman1990, frankgoldwaterfrank2013, perkins2019} a.o.), as identifying clause types is helpful in explaining word order variability and the distribution of missing arguments. There is also reason to believe that learning this basic distinction is necessary for the acquisition of more complex structural properties, such as the semantics of clause-embedding verbs such as \tit{think}, \tit{know} and \tit{wonder}. Learning the clause type distinctions may help children notice the subcategory distinction between these three types of verbs and then could aid in learning related semantic notions like veridicality (\citealt{white2015diss, lewis2017think,dudley2017,hacquardlidz2018}). 
Learning speech act distinctions is crucial not only for for toddlers' language learning (\citealt{ninio1980, hoff1985cds,yoder1994,rowland2003cdswh, valian2003cds, rowe2017wh, gaudreau2021question} among many others), and cognitive development in general (\citealt{hohmann1995educating} among many others). However, figuring out what the relevant clause types and speech acts are is not trivial. 




\section{Challenges in learning clause type and speech act}
\label{sec:intro:challenges}
\subsection{Clause type}
Languages typically employ a wide range of clausal constructions—actives, passives, imperatives, polar and wh-interrogatives, clefts, pseudoclefts, relative clauses, and so on. Nonetheless, many of these distinctions are collapsed from the perspective of clause typing. For instance, verbs that select for declarative complements do not distinguish between actives, passives and clefts. Children, then, need to cluster the right morphosyntactic properties that are relevant for clause types and associate them with the right pragmatic function.


there is the clustering problem. Children must be able to assign a sentence to one of three formal categories. That is, when given a new sentence like (\ref{engcl:cluster}), they should be able to put it in the same formal category as (\ref{engcl:cluster}a) and not (\ref{engcl:cluster}b). 

\bex{engcl:cluster}
Do you want a cookie?
\bxl{}
Is that Bert?
\ex
That’s Bert!
\exl
\eex
Second, there is the labeling problem. Children need to determine the canonical function of each cluster in the system. That is, after clustering sentences into three categories, children still need to learn which one of these clusters is the interrogatives, which is the declaratives, and which is the imperatives. To break into this labeling problem, the learner might need to use the function of the utterance—-the speech act information. If the learner observes that a cluster of sentences is more frequently associated with questions, they might be able to label this cluster of sentences as the one whose canonical function is asking questions (i.e.\ interrogatives), and similarly for declaratives and imperatives. 

\subsection{Speech Act}
The same process applies to speech acts. Children also need to figure out that the utterance they just heard is a question that needs response, and not an assertion. 


- Children need to figure out the conventional effects that certain speech acts have on the current conversation
-- for example, questions require answers, assertions update your belief, requests/commands may add things to your to-do list.
- associate speech act categories with clause types
--- Children need to figure out what speech act is expressed by what kind of clauses with what kind of prosody

\subsection{Many-to-many mapping problem}
Solving the clustering and labeling problem of   

In English, for example, the subject and verb switch their canonical order in interrogatives. But this is not so in all languages. Mandarin particle polar interrogatives like (\ref{ex:m-pol}), for instance, have the same SVO word order as a declarative (\ref{ex:m-dec}); the distinguishing feature is the presence of the sentence-final particle \tit{ma}. Hence, on the formal side, children need to figure out the language-specific syntactic features of their interrogatives. However, as we will see in the next section, the learners might face many challenges along the way.

\bex{ex:m-pol}
\gll Xiayu	le	ma?\\
rain \Asp{} \Sfp{}\\
``Is it raining?''
\eex
\bex{ex:m-dec}
\gll Xiayu	le.\\
rain \Asp{}\\
``It's raining.''
\eex

Learning the syntactic features of interrogatives is not so straightforward. In Mandarin, some phrases characteristically associated with interrogatives also have non-interrogative uses. For example, \twh-phrases can be interpreted as indefinites: sentences containing \twh-phrases can be either questions or assertions:

\bex{ex:m-whamb}
\gll Xiaoxiao mei 	chi 	\tun{shenme} dongxi\\ 
Xiaoxiao \Neg{} 	eat	what	things\\
a.	``What didn’t Xiaoxiao eat?''	\hfill Interrogative \twh\\
b.	``Xiaoxiao didn’t eat anything.''		\hfill Non-interrogative \twh
\eex

\noindent In English, the mapping between word order and clause types is also not without exceptions. For example, subject \twh-interrogatives like \tit{who can come?}~are interrogatives but have an SVO word order like a declarative. 

On the function side, learners face two challenges in associating interrogatives with questions: the first is a general challenge for associating clause types with their canonical function, which is that while clause types constrain the illocutionary force of an utterance, the mapping between them is not inviolable: virtually any clause can be used for any speech act (\citealt{searle1975tax}, \citealt{searle1976class}, \citealt{bachharnish1979}, \citealt{levinson1983},\citealt{searlevanderveken1985}, \citealt{portner2018}, \citealt{starr2014}, \citealt{murraystarr2020} a.o.). For example, polar interrogatives like \tit{Can you pass the salt?}~can be used to issue a request. The second challenge is particular to the mapping between interrogatives and questions. The canonical use of questions is to seek information (\citealt{searle1975tax}, \citealt{levinson1983}, see \citealt{krifka2011q} for a recent overview), but when talking to children, especially pre-linguistic infants, parents can’t genuinely expect them to be in a position to provide that information (\citealt{holzman1972, shatz1978comprehension, tamir1980, yu2019pedagogical}, a.o.). %So how do children figure out the canonical mapping between the formal features of interrogatives and their conventionalized function as questions?  

The same porous mapping problem also exist between clause type categories and speech act categories. Some speech act category can be expressed by more than one clause types, and vice versa. For example, interrogatives can express assertions, questions, requests/commands (\ref{eng-cl:q-all}); and questions can be expressed by \diis{} (\ref{eng-cl:int-all}).
\bex{eng-cl:int-all}
Interrogatives can express all kinds of speech acts:
\bxl Is it snowing? \hfill Question
\ex Aren't you sweet. \hfill Assertion
\ex Can you pass the salt? \hfill Request
\exl
\eex

\bex{eng-cl:q-all}
Questions can be expressed by all kinds of clauses:
\bxl
Is it snowing? \hfill Interrogative
\ex It's snowing? \hfill Declarative
\ex Tell me if it's snowing! \hfill Imperative
\exl
\eex



Despite the variable mappings between sentence form and clause type, and between clause type and conventional function, studies have shown that children seem to distinguish interrogatives and declaratives formally by as early as 12 months (\cite{geffenmintz2015wordorder}), and associate interrogatives with questions by age 3, and possibly as early as 18 months (\citealt{tyack1977, ervintripp1978, berningergarvey1981, rowland2003cdswh, seidl2003wh, casillas2013,casillas2017turn, clark2015turn, lammertink2015turn, gagliardi2016wh, perkins2020filler}, among many others).

Given this success, we ask how children learn to identify interrogatives and questions. 

\section{Pragmatic syntactic bootstrapping hypothesis}
\label{sec:intro:prag-syn-bootstrap}



We hypothesize that children jointly learn the speech act and clause type categories from observations of pragmatic, prosodic, and syntactic features of utterances. On the one hand, children learn to identify questions by exploiting the prosody of the utterance and the pragmatic features of the utterance (such as the social function of the utterance and the social attentional behavior of the speaker); on the other hand, they learn to identify interrogatives by exploiting the syntactic features of the sentence. Crucially, however, learning to identify questions and learning to identify interrogatives are also mutually constrained: children could use speech act information to learn the makeup of interrogative clauses in their language, and use clause type information to learn the pragmatics of questions. 


Previously, this \hypos{} where pragmatics and syntax are mutually constraining has been proposed for the learning of attitude predicates (\citealt{dudleyetal2018, hacquardlidz2018}), modals (\citealt{dieuleveut2021}), and quantifiers (\citealt{knowlton2021}). We argue that the learning of interrogatives and questions is another case where the learner need to exploit the correlation between pragmatics and syntax to succeed.  




%To test this hypothesis, a necessary first step is to establish a systematic, empirical picture on the information contained in the input data that children receive. What clause types are used, and with what function? What features of the context in the input might reveal a questioning act?  To this end, we propose two studies: Study~1 examines speech to English-speaking children and Study~2 examines speech to Mandarin-speaking children. In both studies, we will make use of data from existing corpora of parent-child interactions and annotate each utterance for a theoretically motivated set of features, encompassing a sentence’s syntactic features as well as the corresponding utterance's prosodic and social pragmatic features. In addition to building an annotated dataset of the input, we want to model this learning process computationally to understand \tit{how} children can use the information from their input. Study 3 provides a proof of concept for the \hypos{} for the acquisition of questions and interrogatives.


\section{Discussion and roadmap}
\label{sec:intro:roadmap}
%Although there have been studies on parents' use of speech acts like questions, how children could use to acquire the distinction between declaratives and interrogatives and their association with their conventional functions remains under-explored. 
 
 %By providing data on how parents use questions in child-directed speech, and modeling how children use different signals in this input to learn questions and interrogatives, this project will deepen our understanding of the role of questions in language acquisition and beyond. 
%By systematically examining different aspects of language, from syntax, prosody to pragmatics, and modeling the learning process computationally, this study will provide insights into how children come to learn speech acts and clause types. Additionally, by comparing two languages in which the formal features of interrogativity differ greatly, this research will also advance our understanding of the universal and language-specific problems that children face when learning interrogatives. Finally, the labelled datasets as well as the the time-aligned corpora will be made available to the public for research on children's input beyond the learning of speech acts and clause types. 

 
% Local Variables:
% TeX-engine: xetex
% LaTeX-biblatex-use-Biber: t
% TeX-master: "../main"