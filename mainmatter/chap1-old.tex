% -*- mode: latex; coding: utf-8; fill-column: 72; -*-


\chapter{Introduction}
\label{chap:introduction}

We use language to perform various kinds of speech acts -- providing information, asking questions, making requests, etc. In any language, there are specific signals in the form of a sentence that indicates what speech act it is typically used for. In particular, languages tend to have dedicated clause types for the same three basic speech acts (\citealt{sz1985speechact, konig2007, aikhenvald2016, portner2018}, a.o.): declaratives are typically used for assertions (\ref{ex:intro:intro}a), interrogatives for questions (\ref{ex:intro:intro}b), and imperatives for commands (\ref{ex:intro:intro}c):

\bex{ex:intro:intro}
\bxl
That's Elmo. \hfill Declarative, Assertion
\ex Is that Elmo? \hfill Interrogative, Question
\ex Find Elmo! \hfill Imperative, Request
\exl
\eex

%%%%%%

%%%%%%

However, the surface formal features of these clause types differ greatly from one language to the next. For example, English declarative and interrogative differ in word order: auxiliaries precede subjects in interrogatives (\ref{ex:intro:intro}b) but not in declarative (\ref{ex:intro:intro}a); Mandarin interrogatives and declaratives do not differ in word order, but instead, interrogatives are marked by sentence final particle \tit{ma}, as evident in (\ref{ex:intro:man}a-b): 

\bex{ex:intro:man}
\bxl
\gll Zhe shi Elmo.\\
This is Elmo\\
\trans ``This is Elmo." \hfill Declarative
\ex 
\gll Zhe shi Elmo \tbf{ma}?\\
This is Elmo \Sfp\\
\trans ``Is that Elmo?'' \hfill Interrogative
\ex 
\gll Zhizhi Elmo!\\
Point Elmo\\
\trans ``Point at Elmo!'' \hfill Imperative
\exl
\eex

Therefore, children need to figure out the language-specific surface formal features of their clause types. In this dissertation, I ask how children learn the surface signals associated with the clause types while learning to identify the speech acts that these clauses express. There are several problems that they have to solve.

\section{The problem with clause types}

Clause types are grammatically defined classes of sentences that are closely associated with speech acts (e.g. assertions, questions, requests).\footnote{For some, clause types are form-meaning pairings (\cite{sz1985speechact, ginzburgsag2000interrogative}). In this dissertation, I take clause type as a grammatical concept, sentence mood (or in \textcite['s]{chierchia1990textbook} terminology, sentential force) as the mapping of form and meaning, and speech act as either the act that speaker performs with a sentence, or the conventional effects of a sentence on the discourse. We'll come back to theses theoretical differences in the next chapter.} In the generative tradition, the clause type information of a clause is often analyzed as related to a phonetically null affix or feature occupying $C^{0}$ (\cite{chomsky1995minimalist, cheng1991, rizzi1997, rizzi2001int, chomskylasnik1977,platzack1997imp,akmajian1984clausetype, han1998imp}), e.g. the feature [+int].\footnote{In early formulations, clause type information is represented as a phonetically null Q morpheme as part of S (\cite{katzpostal1964, baker1970int}) in the deep structure of a sentence to solve the meaning preservation problem introduced by the subj-aux inversion rule $T_{q}$ from \textcite{chomsky1957}. For our purposes, this analysis does not differ from the feature analysis, so I will continue use the feature terminology in this dissertation.} 


\section{The problems with speech act}
The function of a sentence can be thought of as the speech act performed with it. For example, Alex's mother uses (\ref{}) to ask Alex's preferences, performing the speech act of questioning.
\bex{}
Do you want to get down? \hfill Alex's mother, Session 01;05;12, Providence Corpus (\cite{ProvidenceCorpus})
\eex

%This paragraph is not good
Speech act theory focuses on what speakers \emph{do} with sentences (\cite{austin1975things, searle1969} a.o.), but this is not the only way to conceptualize speech acts, an issue we will come back to in the next chapter. Regardless of one's theory of speech act, it's generally agreed that the form of a sentence, i.e. its clause type, constrains the types of speech act performed with it. Canonically, declaratives tend to be assertions, interrogatives questions, and imperatives requests. 


But this mapping between them is not inviolable. It is possible that the conventionalized speech act associated with a clause type is not the actual speech act performed with the utterance. Indirect speech acts refer to these mismatching cases where the primary, ``non-literal'' force of an utterance is different from the conventionalized, ``literal'' force of a sentence (\citealt{searle1975tax, searle1976class, bachharnish1979, levinson1983, searlevanderveken1985, portner2018, starr2014, murraystarr2020} a.o.). The common example illustrating this phenomenon is \tit{Can you pass the salt?} as a request. As an interrogative sentence, its conventionalized (and ``literal'') force is questioning, but the primary act performed is requesting. As a result, some speech act categories can be expressed by more than one clause types, and vice versa. For example, interrogatives can express assertions, questions, requests/commands (\ref{eng-cl:q-all}); and questions can be expressed by \diis{} (\ref{eng-cl:int-all}).
\bex{eng-cl:int-all}
Speech acts expressed by interrogatives 
\bxl Is it snowing? \hfill Question
\ex Aren't you sweet. \hfill Assertion
\ex Can you pass the salt? \hfill Request
\exl
\eex

\bex{eng-cl:q-all}
Clause types expressing questions
\bxl
Is it snowing? \hfill Interrogative
\ex It's snowing? \hfill Declarative
\ex Tell me if it's snowing! \hfill Imperative
\exl
\eex


Meanwhile, the learning of speech acts also have the same clustering and labeling problem. Children also need to figure out that the utterance they just heard is a question that needs response, and not an assertion. Clause type might be helpful in discovering the clustering of speech acts, but there might be non-clause type cues as well. %For example, \cite{} explores how facial expressions could be related to speech acts:%


Similar to our discussion in the last section about clause types, solving the clustering problem of speech acts could also be hindered by the many-to-many problem. We've already seen this with the mapping between clause type and speech act, but the same is true for the non-clause type cues. %


Also, these non-linguistic cues could be masked by noise in actual parent-child interactions. For example, in free play settings, parents' eye gaze could be redirected to new stimuli. Similarly, meaningful speech gap could be hard to find due to non-conversation-related distractions.  %this is also not well-phrased

In sum, the learning of speech acts is also subject to the clustering and labeling problem. Clause type information could be helpful in solving the clustering problem, but as the mapping between clause type and speech act is not one-to-one. There might be non-clause type cues in parents' interaction with children, but these cues also suffer from the same many-to-many problem. Moreover, useful patterns in the input could be masked by noise. %So how do children figure out the canonical mapping between the formal features of interrogatives and their conventionalized function as questions? 



\section{Pragmatic syntactic bootstrapping hypothesis}
\label{sec:intro:prag-syn-bootstrap}
Despite all the challenges, infants seem to have figured out clause types and speech acts by 18 months old, as we will see in the next chapter. How do infants learn to identify clause types, speech acts, and their connections, especially with interrogatives and questions?

When discussing learning attitude verbs like \tit{think} and \tit{know}, \textcite{hacquardlidz2018} propose the \hypos{}. They argue that children might be able to use pragmatics and syntax to learn the meaning of attitude verbs. In particular, the pragmatic and syntactic information mutually constrain each other:

\begin{quote}
We propose that what allows the child to infer the lexical semantics of attitude verbs—to which she is never directly exposed—is the exploitation of correlations between syntax (what syntactic complements appear in attitude reports) and pragmatic function (how these attitude reports are used by speakers). The challenges to syntactic and pragmatic bootstrapping resolve each other by being mutually constraining: the syntax helps the learner parse the pragmatic context, while the pragmatics, in turn, helps the learner make use of the syntactic distribution. 

\hspace*{\fill}\hfill \textcite[p.4-5]{hacquardlidz2018}
\end{quote}


This \hypos{} where pragmatics and syntax are mutually constraining has also been proposed for the learning of attitude predicates modals (\citealt{dieuleveut2021}) and quantifiers (\citealt{knowlton2021}). 

The learning of the semantics of attitude verbs is very similar to our current learning problem: learners also need to infer the sentential force which could not be directly observed. %%%why???
Therefore, we could adapt to our current case of learning clause types and speech acts. Specifically, I hypothesis that children learn to identify speech act and clause type, especially questions and interrogatives, in tandem and mutually informative ways: children learn to identify clause types by tracking formal regularities in conjunction with their growing knowledge of speech act and its associated socio-pragmatic cues; similarly, they learn to identify speech acts by tracking socio-pragmatic cues in conjunction with their growing understanding of the morpho-syntactic cues of various clause types.

%On the one hand, children learn to identify speech acts questions by exploiting the prosody of the utterance and the pragmatic cues of the utterance (such as the social function of the utterance and the social attentional behavior of the speaker); on the other hand, they learn to identify interrogatives by exploiting the syntactic features of the sentence. Crucially, however, learning to identify questions and learning to identify interrogatives are also mutually constrained: children could use speech act information to learn the makeup of interrogative clauses in their language, and use clause type information to learn the pragmatics of questions. 

In this dissertation, I test a weaker version of this hypothesis with a focus on clause type identification:

\begin{quote}
Infants learn to identify abstract clause type categories with the speech act information, in conjunction with observations of morpho-syntactic features in the surface form of sentences.
%Infants need to use the speech act information to cluster sentences into the three major clause types. 
\end{quote}

%To test this hypothesis, a necessary first step is to establish a systematic, empirical picture on the information contained in the input data that children receive. What clause types are used, and with what function? What features of the context in the input might reveal a questioning act?  To this end, we propose two studies: Study~1 examines speech to English-speaking children and Study~2 examines speech to Mandarin-speaking children. In both studies, we will make use of data from existing corpora of parent-child interactions and annotate each utterance for a theoretically motivated set of features, encompassing a sentence’s syntactic features as well as the corresponding utterance's prosodic and social pragmatic features. In addition to building an annotated dataset of the input, we want to model this learning process computationally to understand \tit{how} children can use the information from their input. Study 3 provides a proof of concept for the \hypos{} for the acquisition of questions and interrogatives.

To get us one step closer to the full-fledged version of \hypos{}, I also explore what kind of non-clause type cues are present in infants' interaction with parents that will allow them to cluster speech act categories. 




\section{Discussion and roadmap}
\label{sec:intro:roadmap}

Learning to identify clause types and speech acts is important for early language acquisition. The acquisition of various basic syntactic phenomena like argument structure, word meanings, and basic word order, might be aided by an ability to distinguish declarative clauses from other clause types (\citealt{pinker1984, pinker1989, gleitman1990, frankgoldwaterfrank2013, perkins2019} a.o.), as identifying clause types is helpful in explaining word order variability and the distribution of missing arguments. There is also reason to believe that learning this basic distinction is necessary for the acquisition of more complex structural properties, such as the semantics of clause-embedding verbs such as \tit{think}, \tit{know} and \tit{wonder}. Learning the clause type distinctions may help children notice the subcategory distinction between these three types of verbs and then could aid in learning related semantic notions like veridicality (\citealt{white2015diss, lewis2017think,dudley2017,hacquardlidz2018}). As for speech acts, they are not only crucial for toddlers' language learning (\citealt{ninio1980, hoff1985cds,yoder1994,rowland2003cdswh, valian2003cds, rowe2017wh, gaudreau2020q, gaudreau2021question} among many others), but also for cognitive development in general (\citealt{hohmann1995educating} among many others). 

Given the importance of clause types and speech acts, this dissertation investigates how infants come to identify clause types and speech acts. 


This dissertation is organized as follows. Chapter~\ref{chap:background} examines the developmental trajectory of speech acts and clause types, especially questions and interrogatives. As we will see, English-acquiring infants as early as 18 months seem to have already sensitive to the distinctions between different clause types and speech acts, and seem to understand the mapping between questions and interrogatives. The same holds for infants acquiring other languages as well, even though we have less evidence. Our question then is, how do 18-month-olds learn to figure out clause types?

Chapter~\ref{chap:eng-cl} looks at how English-acquiring 18-month-olds could have solved the problem. Specifically, is information from syntax enough for children to find the right three clause type categories, or do they need pragmatic information like the speech act of the sentence to find the right clustering? I build two computational models to address this question, a \distlearner{} (\dlearnerabbr{}), and a \praglearner{} (\plearnerabbr{}). These two learners both need to infer the abstract clause type, but \dlearnerabbr{} draws inferences from syntactic information alone while \plearnerabbr{} uses both syntactic and pragmatic information. I use a corpus study to first provide a quantitative description of the type of input that infants get, and use the resulted annotated dataset as input for the computational models. I find that pragmatic information is indeed important for solving the clustering problem: without the speech act information, \dlearnerabbr{} cannot find the right clause types. Additionally, a little pragmatics goes a long way, as  even if 80\% of the pragmatic information is noise, it still improves the learner's performance. 

Chapter~\ref{chap:man-cl} applies the same methodology to another language, Mandarin. Mandarin-acquiring infants figure out the clause types of their language around the same age as English-acquiring infants, but the two languages employ different morpho-syntactic features for clause typing. How do Mandarin-acquiring infants solve the problem? Do they also need pragmatic information? I compare the same two learners, and found that pragmatics information is crucial for identifying Mandarin clause types as well.

But so far, we are operating under the assumption that infants have speech act at their disposal. How do they figure out the speech acts of parents' utterances? Of course their knowledge of clause type might help, but are there signals from other sources? Chapter~\ref{chap:eng-sp} explores cues from prosody and parents' behavior that might help infants identify questions. Chapter~\ref{chap:discussion} concludes the dissertation.
 
% Local Variables:
% TeX-engine: xetex
% LaTeX-biblatex-use-Biber: t
% TeX-master: "../main"