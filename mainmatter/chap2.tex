\chapter{Background}
\label{chap:background}


\section{Theories of speech act and clause type} \label{sec:bg:theory}

\subsection{Clause type}
\label{sec:bg:clause}
If we think of clause type information as encoded in $C^{0}$ (for example, as a [$+/-$ int] feature,  \cite{chomskylasnik1977,rizzi1997}; cf. \cite{rizzi2001int}), what children need to figure out is which sentences have the [+int] value and which ones take the [-int] value. In English, the [+int] value of $C^{0}$ is associated with the syntactic rule of subject-aux inversion. So the child must come to treat (\ref{engcl:cluster}) and (\ref{engcl:cluster}a) alike (because of subject-aux inversion triggered by [+int] in $C^{0}$) and distinct from (\ref{engcl:cluster}b).%\footnote{} 


\bex{engcl:cluster}
Do you want a cookie?
\bxl{}
Is that Bert?
\ex
That’s Bert!
\exl
\eex

This knowledge, that the presence of subject-auxiliary inversion indicates a [+int] and not [-int] value of $C^{0}$, has to be learned from the input and likewise for other surface formal properties relevant for clause typing. This is because the relevant properties indicative of the value of $C^{0}$ (and the clause type category of the sentence) differ from language to language. For example, as we will discuss in detail in the next chapter, Mandarin-acquiring infants, unlike English-acquiring infants, need to pick out that [+int] is expressed by the presence of a sentence-final particle, specifically the particle \tit{ma}, and hence should use this surface feature to cluster sentences into clause types.


\subsection{Speech act} \label{sec:bg:speech}
Frege observed that it is possible to express a thought without judging it to be true, and that considering a particular claim and judging it to be true are different things altogether. In asking a polar question like \tit{Is it raining? }one is not judging anything to be true. But in answering \tit{Yes }or asserting the declarative sentence like \tit{It is raining,} one does. Nevertheless, \tit{Is it raining?} and \tit{It is} raining have something in common: they have the same content, or as Frege puts it, they contain the same \tit{thought}. From this observation he derives the influential distinction between \tit{content} and \tit{force}. The following passage from \tit{Der Gedanke} makes this distinction clear:

\begin{quote}
    

An interrogative sentence and an assertoric one contain the same thought; but the assertoric sentence contains something else as well, namely assertion. The interrogative sentence contains something more too, namely a request. Therefore two things must be distinguished in an assertoric sentence: the content, which it has in common with the corresponding propositional question; and assertion. The former is the thought or at least contains the thought. So it is possible to express a thought without laying it down as true. The two things are so closely joined in an assertoric sentence that it is easy to overlook their separability. 

$\,$\hfill Frege 1918: 62, Translated by Peter Geach and R. H. Stoothoff
\end{quote}

That is, the polar question \tit{Is it raining?} and the assertion \tit{It is raining} both contain the same thought, namely \tit{whether it is raining.}  But \tit{Is it raining?} presents this thought with questioning force and \tit{It is raining} presents it with assertoric force. By presenting a thought with assertoric force one expresses that one judges the thought to be true. But there are other ways of presenting the same thought, for example with questioning force. \footnote{However, Frege argues that since the question of truth does not arise for sentences expressing commands, requests, wishes etc., so even though they have sense, their sense is not a \tit{thought}.}

The idea of force is further developed by Austin in his work on speech acts. The core observation is that utterances may have a variety of forces, not just asserting and questioning, but also promising, declaring, warning, marrying, naming and so on. According to Austin, the same sentence can in principle be used with a broad variety of forces:

\begin{quote}
    
We may be quite clear what \tit{Shut the door} means, but not yet at all clear on the further point as to whether as uttered at a certain time it was an order, an entreaty or whatnot. What we need besides the old doctrine about meanings is a new doctrine about all the possible forces of utterances.

\hfill Austin 1956/1979: 251
\end{quote}

This new doctrine is to categorize utterances primarily by what one \tit{does} with them, rather than what they \tit{mean.} His analysis of what one does with an utterance happens on three levels: the locutionary act (the act of speaking), the illocutionary act (the act done through speaking, e.g. informing, warning, requesting), and the perlocutionary act (the by-product of producing the utterance). Let’s walk through an example with these three levels. If we utter a sentence like \tit{The campus is closed tomorrow}, the locutionary act is the speaker utters these very words; the illocutionary act is that the speaker informs the addressee of the news about campus shutdown, and (at least one of) the perlocutionary acts here is that the speaker is changing the addressee’s plans about coming to campus tomorrow. The illocutionary act is of particular interest, as it is the speaker’s intent to inform by producing the utterance. To be more precise, according to Austin, the speaker and hearer are engaging in a conventional procedure of information sharing in which performing a certain locutionary act \tit{is} to perform a certain illocutionary act. The locutionary act has its illocutionary force in virtue of being a part of this conventional procedure. The utterance has informing force by convention.

Austin is in particular interested in developing a taxonomy of speech acts. He focusses his attention on categorizing the \tit{speech act verbs}: the verbs that can be explicitly used to invoke a particular convention by using the word \tit{hereby}. For example, one invokes the procedure of information sharing (i.e. one makes explicit that one is performing the illocutionary act of informing) by saying \tit{I hereby inform you}. Not all verbs with a roughly communicative meaning are speech act verbs. For example, it is odd to say \tit{I hereby surprise you.} So, \tit{surprising} is not an illocutionary force.

Searle (1969) points out that there is no reason to believe that the verbs of English (or any other language, for that matter) exhaust the illocutionary forces. There could be illocutionary forces that do not correspond to a verb. However, Searle agrees with Austin that speech acts, and in particular their illocutionary forces, involve certain social conventions and constitutive rules. Searle moreover follows Grice (1957meaning) in stressing the role of the speaker’s intention to perform a certain act and to be recognized as performing this act. What makes the intention recognizable, according to Searle, is how the locutionary act is in accordance to certain conventional rules:

\begin{quote}
    

In the performance of an illocutionary act, the speaker intends to produce a certain effect by means of getting the hearer to recognize his intention to produce that effect, and furthermore, if he is using words literally, he intends this recognition to be achieved in virtue of the fact that the rules for using the expressions he utters associate the expressions with the production of that effect. \hfill (Searle 1965, 259)
\end{quote}

Searle (1975, et seq.) then gives the following broad categorization of speech acts by what they are used to achieve.

\bex{spact:searle-tax}
\bxl
assertives = speech acts that commit a speaker to the truth of the expressed proposition
\ex directives = speech acts that are to cause the hearer to take a particular action, e.g. requests, commands and advice
\ex commissives = speech acts that commit a speaker to some future action, e.g. promises and oaths
\ex expressives = speech acts that express on the speaker's attitudes and emotions towards the proposition, e.g. congratulations, excuses and thanks
\ex declarations = speech acts that change the reality in accord with the proposition of the declaration, e.g. baptisms, pronouncing someone guilty or pronouncing someone husband and wife
\exl
\eex

Like Frege, Searle also draws a strict distinction between force and content. However, there are some significant differences between Frege’s and Searle’s views on the distinction. For Frege, content is what the sentence radical expresses (e.g. what’s in common between a declarative and a polar interrogative). But for Searle expressing a content is an act and he does not see how “sentences could perform acts” (rather, acts are performed by speakers, p. 257). Thus, speakers perform illocutionary acts and many illocutions involve the expression of a content.\footnote{Searle explicitly mentions that there are illocutionary acts without contents, such as \tit{Ouch} or \tit{Hurrah!}} The content of an illocutionary act is the proposition that the speaker expresses by uttering a particular sentence. Nevertheless, of course, which sentence the speaker utters (and how) influences what the content and the force of the illocutionary act is. According to Searle, sentences have a proposition-indicating element and a function-indicating device also called, in later work, the illocutionary force indicating device (IFID) of a sentence (Searle 1975). For Searle, IFIDs are formal features including, in English, word order, stress, intonational contour, punctuation, the mood of the verb, and certain lexical elements (notably, Austin’s speech act verbs). However, he also notes that in many circumstances the context alone is sufficient to indicate force, so IFIDs do not always need to be explicit (p. 257). He concludes that there are rules for indicating propositions and rules for indicating force, but that these rules can be investigated independently from one another. 

However, if we taxonomize illocutionary acts by the formal features of the utterances with which they are performed, we encounter the phenomenon of indirect speech acts. Indirect speech acts will play a major role later in this dissertation. It is important to note that it only makes sense to speak of \emph{indirect} speech acts once a taxonomy is in place. In the case of Searle’s taxonomy of speech acts by their associated IFIDs, an indirect speech act is an illocutionary act that has the IFIDs of another illocutionary act. For example, \tit{Can you pass me the salt?} uttered at the dinner table is conventionally understood to perform an act of requesting, but it has the formal features of a question. \textcite{searle1975indirect} insists that such utterances still have the force indicated by their formal features and, thus, that Can you pass me the salt? has the illocutionary force of a question. Nevertheless, Searle claims, one can perform a request by performing a question. In this example, the questioning force is the “secondary” illocutionary act and also the ``literal” meaning of the utterance, whereas and the requesting force is “primary” and “nonliteral” (170). This however breaks with the idea that illocutionary forces are indicated by IFIDs, for the utterance contains no IFID indicating requesting. Although, in the absence of explicit IFIDs the context can fill in the gaps, the theory does not provide for the context to change the IFIDs of an utterance. 

\textcite{searlevanderveken1985} then pursue the project of developing a taxonomy for all possible forces (i.e. types of illocutionary acts). They suggest that each possible illocutionary force can be defined as a septuple of values, each of which is a “setting” of a value within one of the seven characteristics of illocutionary acts: the illocutionary point, its degree of strength, the mode of achievement, the content conditions, the preparatory conditions, the sincerity condition, and the degree of strength of the sincerity condition. It follows, according to this suggestion, that two illocutionary forces F1 and F2 are identical just in case they are characterized by the same setting of these seven values.

However, as any group of annotators who attempt to label corpus data using any of the Austinian-Searlean methods for defining and determining speech act types will tell you, such definitions are extremely finicky. Consequently, achieving any sort of inter-annotator agreement is next to an impossible task. This problem already manifests in some way in our previous example about the utterance \tit{the campus is closed}. The formal features of the utterance are insufficient to determine the speech act that is being performed. When talking about the example, it seems natural to say that the illocutionary act is informing, but couldn’t we say just as well that the speaker intends to produce an effect of warning with the sentence? Or an effect of lamenting? Without an explicit speech act marker like \tit{inform}, \tit{warn} or \tit{lament} (e.g. in \tit{I’m warning you}), it’s almost impossible to discern these effects, especially as a corpus annotator. The possibility of indirect speech acts exacerbates this issue further, as \tit{the campus is closed} could also be an indirect answer to a question or, indeed, a request to hand over a key, or extend a homework deadline.\footnote{An unfortunate consequence of the multitude of illocutionary forces that one may attribute to the same utterance is that, in order to boost inter-annotator agreement, annotation schemas sometimes blend formal categories with functional categories. For example in the schemas discussed in \textcite{ninio1994} and \textcite{dialogact}, \twh-interrogative and polar interrogatives are included in the labeling.}

At this point, it appears fruitful to attend not to the fine distinctions of Austinian speech act verbs, nor to the distinctions of Searle and Vanderveken’s metaphysics of action, but rather to how utterances affect the context they are made in (\cite{hamblin1971, stalnaker1978, lewis1979scorekeeping, gazdar1981speech}). As \textcite{chierchia1990textbook} points out, while a sentence like (\ref{spact:chierchia}) can be uttered to perform many different speech acts, such as claiming, guessing, reminding, warning, or threatening, there is a similarity across all these cases. Namely, that a declarative sentence S places the proposition expressed by S “in the common ground and discards any possibilities rendered no longer live because of their inconsistency with that (possibly new) information” (\cite[p.171]{ chierchia1990textbook}). 

\bex{spact:chierchia}
The bull is in the field.
\eex

This way of conceptualizing speech acts not in terms of the rules governing their use, but in terms of their discourse effect is sometimes referred to as the “discourse dynamics tradition” (\cite{murraystarr2020}). 

Before we dive into this tradition, we need to have a quick note on the terminology, as sometimes the same term refers to different notions, both within the discourse dynamics tradition and compare to the classical speech act theories. 

\textcite{portner2018} is the only one explicitly defines what he means by clause type, sentence mood, and sentential force: 
\bex{portner:concepts}
\bxl
\tbf{Clause types} are grammatically defined classes of sentences which correspond closely with sentence moods. (p.122)
\ex\tbf{Sentence mood} is an aspect of linguistic form conventionally linked to the fundamental conversational functions within semantic/pragmatic theory; (p.122) sentence mood is fundamentally a matter of which syntactic features a clause has (p.141). 
\ex \tbf{Sentential forces} are the fundamental conversational functions with which sentence moods are associated. (p.124)
\exl
\eex



\section{Acquisition of speech acts and clause types}
\subsection{Knowledge about speech acts}

Learning communicative acts
•	Csibra (2010)
o	Infants are great at communicative intent (inferring that they are being talked to
o	Gaze, motherese (name calling) and joint responsivity (turn-taking) are the main factors 
o	They learn informative intent (speaker goals) by using communicative intent. For example, they can generalize beyond singular instances when the instructor is communicative. 
•	Bates et a; (1975): 
o	Children (2mo-16mo); children above 10mo produce proto-imperatives/declaratives
o	Perlocutionary communication: children cry-get desired results
•	Behne et al. 2015: 
o	9mo can distinguish people who is unwilling/unable to hand them a toy 
•	Sakkalou et al. 2013: 
o	13-14mo can distinguish immediate goal-directed and intentional actions and ignore actions that are accidental
•	Warneken and Tomasello 2007:
o	14mo recognize other people's goals and if things go wrong, children offer to help achieving people's goals
•	Woodward (1998, 1999): 
o	6-9mo can recognize goals
o	If a researcher keeps reaching for a toy, the child gets surprised if the person reaches for another toy regardless of location
•	Tomasello (1995)
o	9-12mo can use jointattention 
o	Later: coordinated joint interaction
•	Scott et al. (2013):
o	Parents following-in on children's attention
o	Children following-in on parent's attention
o	Both predict vocab size at 14-18mo
•	Csibra: Natural Pedagogy 
o	Children are born with the abiilty to recognize ostensive communicative cues
o	Senju and Csibra (2008): 6.5mo direct attention to the object that is attended by another speaker, but only if the speaker engage with them in a joint interaction (talk to the kid or look at the kid first)
•	O'Neil (1996):
o	2yo take into account what adults know when requesting help from adults
•	Moll and colleagues (2008)
o	14mo can keep track of experience shared with others
o	Experimenter 1 and the kid interact, always show excitement toward object 1
o	When later asked "Give me it!" child gives Experiment 1 the exciting ball, but if Experiment 2 asks, less so
•	Brinton et al (1986)
o	Children do not stop to repair misunderstanding until 9yo


\subsection{Interrogatives and questions in early language acquisition}\label{bg:acq}
	
\noindent Based on the observations outlined in the previous section, the task of acquiring the association of interrogative clauses with questions may not be straightforward. However, studies have shown that English-speaking children are sensitive to the syntactic and prosodic features of interrogatives:  around 12 months, English-speaking infants show sensitivity to differences in word order (\citealt{geffenmintz2015wordorder}) and the rising and falling prosodic contours (\citealt{soderstrom2011,Hay2019}) associated with declaratives and polar interrogatives respectively. By~2 years, children may be able to associate interrogatives with questions: they predict a change of speaker when they hear interrogative sentences and not declaratives (\citealt{casillas2013, casillas2017turn}), and use word order information to predict upcoming turn changes (\citealt{casillas2013, casillas2017turn,lammertink2015turn}), suggesting that they might expect interrogatives to be associated with questions, which need to be answered.

Using corpus data, many studies have shown that English-speaking children start producing interrogatives around 20 months starting with polar and \twh-interrogatives (\citealt{tyack1977, stromswold1995, rowland2003cdswh}). From the comprehension side, results from corpus studies on parent-child interactions show that children begin to respond to parents’ questions appropriately around one and a half years old (\citealt{ervintripp1978, steffensen1978, shatz1978comprehension, shatz1978communicative, berningergarvey1981, shatzmccloskey1984, clark2015turn, moradlou2020} among others). Data from preferential looking tasks show that already at 15 months old, infants look at the objects corresponding to the answers of \twh-interrogatives (\citealt{seidl2003wh, gagliardi2016wh, perkins2020filler}), though their success might not necessarily reflect knowledge of \twh-interrogatives, as they might be relying on their knowledge of verb argument structure in the task (\citealt{perkins2019}).

Mandarin-speaking children are observed to produce \ma-interrogatives, A-not-A interrogatives, and \twh-interrogatives around 2 years old (\citealt{miao1986acq, miao1992, lee1989acq, litang1991int, lichen1997compprod, lichen1997comp, fan2012, lijingwong2017}). Experimental studies testing children’s comprehension suggest that Mandarin-children have adult-like interpretation of \twh-interrogatives around 3 (\citealt{fahn2003acq}), but not of polar interrogatives (\citealt{moradlou2020}). 

In our prior work, we find that Mandarin-speaking three-year-olds have adult-like interpretations of \twh-phrases, of both the interrogative and the non-interrogative uses. Our first experiment tested sentences like (\ref{bg-acq:dou}) and (\ref{bg-acq:ba}), where the interrogative and non-interrogative interpretation of the \twh-phrase is disambiguated by morpho-syntactic information (the presence/absence of the particle \dou{}):

	
\bex{bg-acq:dou}
\gll Xiaoyang 	\tun{shenme} 	\tit{dou} 	fangzai 	xiangzili 	le.\\
Lamb	what	\Dou{}	put 	box		\Asp{}\\
``Lamb put everything in the box.''
\eex
\bex{bg-acq:ba}
\gll Xiaoyang	\tit{ba} \tun{shenme} 	fangzai 	xiangzili 	le\\
	Lamb	\tsc{ba}	what	put	box		\Asp{}\\
	``What did Lamb put in the box?''
\eex


Our results show that children, like adults, respond with “yes/no” to sentences like (\ref{bg-acq:dou}) but not to (\ref{bg-acq:ba}), suggesting that they can treat the former as an assertion and the latter as a \twh-question. In a second experiment, we present 3-year-olds and adults with negated sentences like (\ref{bg-acq:negwh}) where the sentence is ambiguous between an assertion and a \twh-question. In this case, the two interpretations are disambiguated by prosodic cues: assigning prosodic prominence on \tit{shenme} gives rise to a \twh-question interpretation, and without prosodic prominence, the sentence is an assertion. Our results again show that 3-year-olds, like adults, can access both interpretations of the sentence.


\bex{bg-acq:negwh}
\gll Xiaoyang mei	fang	\tun{shenme} shuiguo zai	xiangzili\\
Lamb \Neg{}	put	what		fruit		in	box\\
a. ``What fruit didn't Lamb put in the box?''\\
b.	``Lamb didn't put any fruits in the box.''
\eex

\noindent In summary, children show sensitivity to the formal properties of interrogatives and the link between interrogatives and questions by age 3, and possibly much earlier. This project will lay the empirical foundation for answering the question of how they arrive at this knowledge by looking at the kind of input children receive. Previous studies on questions in speech to children suggest that parents use questions in many different ways (\citealt{holzman1972, shatz1979, tamir1980, yu2019pedagogical}). In particular, the frequency of pedagogical questions discussed in the last subsection is higher in child-directed speech than in adult-directed speech (\citealt{stivers2010}). Prior work in our lab (\citealt{zaitsu2020}) investigates the frequencies of the three basic clause types and the corresponding speech acts in speech to children between the age 1 to 3 from the Providence corpus (\citealt{ProvidenceCorpus}) of CHILDES (\citealt{CHILDES}). The results show a relatively robust association of declaratives with assertions, but less so of interrogatives and questions due to questions being often asked via rising declaratives, and requests made with interrogatives. To determine what kind of information bridges this gap, this project focuses on interrogatives and questions, and looks at a combination of syntactic, prosodic, and pragmatic features.




\section{Speech acts and clause types in English}

	\subsection{Pragmatics of questions}\label{bg:prag}
Learners of both (perhaps all) languages face a further difficulty. Besides their main function of eliciting information, interrogative clauses can also be used to fulfill a range of other conventional functions. Test questions like (\ref{bg-prag:test}) do not present the asker as lacking information; parents can use questions like (\ref{bg-prag:ped}) to teach children object labels; indirect requests like (\ref{bg-prag:req}) are not genuinely soliciting information; and rhetorical questions act similar to assertions (\ref{bg-prag:rhe}). Such uses of interrogatives might mask the association of interrogatives with information-seeking speech acts. Additionally, other clause types like declaratives can be used as questions (e.g. rising declaratives, \citealt{gunlogson2004,gunlogson2008,jeong2018,rudin2018}). All these are potential sources of noise in the input with respect to the mapping between form and function.

\vspace{-2ex} 
\noindent
\begin{minipage}[t]{0.4\linewidth}	
\bex{bg-prag:types}
\bxl\label{bg-prag:test}
What is H$_{2}$O?		\hfill \tit{Test}
\ex \label{bg-prag:ped}
What's this?	\hfill \tit{Pedagogy}
\exl
\eex
\end{minipage}
\hspace{1cm}
\begin{minipage}[t]{0.45\linewidth}
\bex{}
\bxl\label{bg-prag:req}
Can you pass the salt?			\hfill \tit{Request}
\ex \label{bg-prag:rhe}
Are you crazy?	\hfill \tit{Rhetorical}
\exl
\eex
\end{minipage}
%\vspace{-1ex}


2.6-year-olds can use the presence/absence of verb morphology to tell whether the sentence is a pro-drop declarative or an imperative (\cite{orfitellihyams2012subj}). 

\noindent However, common to all the functions listed above is the fact that in a conversation, questions expect responses (\citealt{duncan1972turn}). In turn-taking theory, questions usually mark the turn-transition points: after a question, the current speaker yields their turn and appoint the next speaker while the other participants of the conversation would pick up the turn by answering the question (\citealt{kendon1967gaze, argyle1972gaze, levinson1983, tice2011turn} among others, see \citealt{enfield2010} for a survey).  Additionally, questions are argued to set up the topics and issues in a conversation (\citealt{roberts2012,farkasbruce2010}), and speakers could use questions to keep track of where they are in a conversation. Therefore, despite the fact that questions can be used to fulfill many functions, the role of questions in a conversation seems to be clear: speakers use questions to solicit responses and set up topics for discussion. If children are sensitive to signals of response expectations or topic settings, they could make use of that information to identify questions, and, in turn, interrogatives. 
%As mentioned above, combining syntactic and prosodic information may explain some mismatches, as with the case of rising declaratives. Other syntactic markers might be helpful as well: many have noted that interrogatives that function like imperatives tend to take specific forms such as having modals like (\ref{bg-prag:req}), or \tit{why don't} and \tit{why not} interrogatives (i.e. \tit{whimperatives}, \citealt{sadock1974, green1975whimp}).

In summary, children might face many challenges when learning interrogatives and their mapping to questions: English-speaking children should identify word order as a cue for interrogativity, but this information might be buried under many exceptions; Mandarin-speaking children should identify the markers for interrogativity (e.g.~\twh{}) despite some of these markers occurring in non-interrogative environments. Moreover, in both languages, children have to recognize the link between interrogativity and questionhood, which might be masked by exceptions in their input. So, how do children acquire interrogatives and their relationship to questions? In the next section, we will see that despite the noisiness of the input data, there is evidence suggesting that children nevertheless have adult-like understanding of interrogatives, questions, and their relationship at around three years old if not younger. 

\subsection{Interrogatives in English and Mandarin}\label{bg:syn}
	
\noindent While all languages have interrogative clauses, the formal features associated with this clause type vary. 
\noindent Mandarin interrogatives have different formal features than English interrogatives. For \twh-interrogatives, Mandarin \twh-phrases (\citealt{huang1982, cheng1991} among many others) do not need to be fronted to clause-initial position. Compare the declarative sentence in (\ref{bg-syn:dec-man}) with the \twh-interrogative in (\ref{bg-syn:wh-man}), the \twh-phrase \tit{shenme} occurs in the same position in the interrogative (\ref{bg-syn:wh-man}) as the noun phrase \tit{zaocan} in (\ref{bg-syn:dec-man}). 


\vspace{-1.5ex} 
\noindent
\begin{minipage}[t]{0.45\linewidth}	
\bex{bg-syn:dec-man}
\gll Xiaoxiao	chi-le zaocan.\\
Xiaoxiao eat-\Asp{} breakfast \\
``Xiaoxiao ate breakfast.''
\eex
\end{minipage}
\hspace{0.5cm} %%这中间不能空行,否则不能compile
\begin{minipage}[t]{0.5\linewidth} 
\bex{bg-syn:wh-man}
\gll Xiaoxiao	chi-le \tun{shenme}.\\
Xiaoxiao eat-\Asp{} what \\
``What did Xiaoxiao eat?''
\eex
\end{minipage}
\vspace{0.5ex}

\noindent 
Polar interrogatives in Mandarin also have SVO word order, but are distinguished from declaratives by surface features like the question-forming particle ma or the A-not-A construction, as in (\ref{bg-syn:ma}) and (\ref{bg-syn:anota}).

\vspace{-2ex} 
\noindent
\begin{minipage}[t]{0.45\linewidth}	
\bex{bg-syn:ma}
\gll Xiaoxiao	chi-le	zaocan		\tun{ma}\\
Xiaoxiao	eat-ASP	breakfast	\Sfp{}\\
``Did Xiaoxiao eat breakfast?''
\eex
\end{minipage}
\hspace{0.5cm} %%这中间不能空行,否则不能compile
\begin{minipage}[t]{0.5\linewidth} 
\bex{bg-syn:anota}
\gll Xiaoxiao	\tun{chi-mei-chi}	zaocan?\\
	Xiaoxiao	eat-\Neg-eat	breakfast\\
	``Did Xiaoxiao eat breakfast?''
\eex
\end{minipage}
\vspace{1ex}

\noindent Sentence final particles (SFPs) like \ma{} in Mandarin are optional particles at the right edge of a clause (\citealt{chao1968, zhudexi, huang1982, cheng1991, liboya2006} among others). Many SFPs can occur in any clause type, but \ma{} can only occur in polar interrogatives that do not have the A-not-A form seen in (\ref{bg-syn:anota}). Because of these differences, Mandarin-speaking and English-speaking children must pursue different strategies: English learners might mostly rely on word order information to distinguish interrogatives and declaratives, and children learning Mandarin might rely on interrogative phrases like \tit{ma}. 

However, Mandarin learners have to contend with an additional difficulty, namely some of these phrases are not uniquely associated with interrogativity. Two prominent examples are \twh-phrases, which can have interrogative and non-interrogative interpretations in Mandarin, and the question particle \ma, which has a homophone that can occur in non-interrogative clauses.

\vspace{-2ex}
\noindent 
\begin{minipage}[t]{0.45\linewidth}	
\bex{bg-syn:ambwh}
\gll Xiaoxiao	mei	chi	\tun{shenme}\\
Xiaoxiao	\Neg{}	eat	what\\
a.	What didn't Xiaoxiao eat?\\
b.	Xiaoxiao didn't eat anything.\\
\eex
\end{minipage}
\hspace{0.5cm} %%这中间不能空行,否则不能compile
\begin{minipage}[t]{0.5\linewidth} 
\bex{bg-syn:ambma}
\gll Xiaoxiao	chi-le		zaocan		ma\\
	Xiaoxiao	eat-\Asp{}	breakfast	\Sfp{}\\
a.	Did Xiaoxiao eat breakfast?\\
b.	(Obviously) Xiaoxiao ate breakfast!\\
\eex
\end{minipage}
\vspace{-1.5ex}

\noindent 
The sentences in (\ref{bg-syn:ambwh}) and (\ref{bg-syn:ambma}) can be read as either questions or assertions. Thus, observing the \twh{} \tit{shenme} or sentence final \ma{} is not sufficient for a learner to determine the clause type. Learners must rely on additional cues from the context or prosody to determine the correct interpretation. In (\ref{bg-syn:ambwh}), when the context is compatible with both readings, assigning prosodic prominence (normally associated with focus) to the \twh-phrase is indicative of the interpretation as a question; in (\ref{bg-syn:ambma}), the question reading is associated with rising intonation, whereas the assertion reading is associated with falling intonation. 

	%\begin{comment}
		\subsubsection{Prosodic Features}\label{bg-feature:pro}
\noindent The precise relationship between prosodic features, clause type, and speech act may not be so straightforward. Cross-linguistically, rising contour is usually associated with a questioning act (\citealt{bolinger1978, ladd1981, gussenhovenchen2000, ladd2001typology}, a.o.). In English, declaratives tend to be associated with falling intonation, polar interrogatives tend to bear rising intonation, while \twh-interrogatives also generally bear a falling contour (\citealt{ladd1981, hedberg2014corpus}). When declaratives are associated with rising intonation, they tend to be interpreted as questions (i.e. rising declaratives, \citealt{ladd1981,gunlogson2004,gunlogson2008,jeong2018,rudin2018}).  

In Mandarin, experimental studies controlled for lexical tone find that compared to assertions, in questions, the F0 of the entire utterance is raised (\citealt{ho1977intonation, shen1991question, yuan2002prosody, nikawai2004, lee2005prosody, liuxu2005, yuan2006qprosody, yuan2011perception, liufang2009prosody, jiangchen2011} among many others), but different types of interrogatives might end with different contours: \ma-interrogatives and rising declaratives tend to have a final rise contour (\citealt{shen1991question, zeng2004tones}), and A-not-A and \twh-interrogatives tend to end on falling intonation (\citealt{shen1991question, nikawai2004}, cf. \citealt{liuxu2005,yang2020wh}). Perception experiments show that speakers tend to use the overall F0 of the whole sentence to distinguish questions from assertions (\citealt{jiangchen2011, gryllia2020typing}), and it is more difficult to use clause-final prosodic cues, especially if the final syllable has a rising tone (\citealt{yuan2011perception}). 

To summarize, in both languages, prosodic information alone may not be sufficient to predict the clause type of a sentence. But prosodic information may help explain some of the mismatches between clause type and speech act, as declaratives that function as questions might be associated with rising intonation.
%\end{comment}
	


