\chapter{Conclusion and discussion}
\label{chap:discussion}

\section{Summary of findings}
This dissertation examines the acquisition of clause type and speech act by English- and Mandarin-acquiring children. Languages tend to have three major clause types (declaratives, interrogatives, imperatives), dedicated to three main speech acts (assertions, questions, commands, \cite{sz1985speechact} among others). However, the particular forms that these clause types take differ from language to language, and have to be learned. Previous experimental results suggest that by 18 months old, children differentiate these clause types and associate them with their canonical speech act (\cite{geffenmintz2011,geffenmintz2015wordorder,casillas2017turn,perkins2019,marshmallowqueen}). To gain this ability, children need to identify the right categories of clauses (the \tbf{clustering problem}) and figure out what speech act they are canonically used for (the \tbf{labeling problem}). 

This dissertation investigates the extent to which learners need to rely on pragmatic information (i.e., knowing what speech act a given utterance of sentence is conveying), to solve not just labeling, but the clustering itself. We examine the role of pragmatics computationally by building two Bayesian clustering models. We find (a) that a learner must have access to some pragmatic information in order to find the right clause types but (b) this learner can succeed with very limited access to pragmatic information. 

Chapter~\ref{chap:eng-cl} looks at how English-acquiring 18-month-olds could solve the problem. Specifically, is information from syntax enough for children to find the right three clause type categories, or do they need pragmatic information like the speech act of the sentence to find the right clustering? I build two computational models to address this question, a \distlearner{} (\dlearnerabbr{}), and a \praglearner{} (\plearnerabbr{}). These two learners both need to infer the abstract clause type, but \dlearnerabbr{} draws inferences from syntactic information alone while \plearnerabbr{} uses both syntactic and pragmatic information. I use a corpus study to first provide a quantitative description of the type of input that infants get, and use the resulted annotated dataset as input for the computational models. I find that pragmatic information is indeed important for solving the clustering problem: without the speech act information, \dlearnerabbr{} cannot find the right clause types. Additionally, a little pragmatics goes a long way, as  even if 80\% of the pragmatic information is noise, it still improves the learner's performance. 

In Chapter~\ref{chap:man-cl}, I apply the same methodology to another language, Mandarin. Mandarin-acquiring infants figure out the clause types of their language around the same age as English-acquiring infants, but the two languages employ different morpho-syntactic features for clause typing. How do Mandarin-acquiring infants solve the problem? Do they also need pragmatic information? I compare the same two learners, and found that pragmatics information is crucial for identifying Mandarin clause types; without pragmatic information, learners might not be able to identify any of the clause types correctly.

But, as said, there is a danger of circularity.
Our \hypos{} assumes that children can infer the speech act categories of the sentence at this point. While some evidence suggests that 18-month-olds can infer speech acts, their inferences might not be perfect. Consequently, they might have only limited access to speech act information. Moreover, there's the problem of how children can infer speech act categories in the first place -- and it is undeniable that the clause type information is useful for solving this problem. As adults, the primary way we identify the act performed by a given sentence is through its clause type. But this is precisely the problem that the child is trying to solve (i.e. identifying the clause type). If children need speech act information to identify clause type categories, but they also need clause type information to identify speech act categories, it seems that we have a chicken-and-egg problem. 

We can break out of the circularity by observing that, to get the learning process going, children do not need \emph{perfect} speech act information. I will evaluate the \hypos{} by 
%I address the first problem by
simulating the learning of clause type with various degrees of noise in the speech act information, so that we can see how much pragmatics a learner needs to succeed at the clustering problem. I then tackle the question of where even imperfect speech act information could come from, if no clause type information is accessible,
%address the second problem
by exploring non-clause type cues for speech act information in the input. I find that the prosody of parents' utterances, speech gap between utterances, and direct eye gaze might be useful in learning the distinctions between speech acts.
These are noisy cues for speech act information, but cues nonetheless. As my simulations show that even very noisy speech act information is useful for learning clause types, this suggests a way out of the circularity. Learners can make use of non-clause type cues towards speech act, which may be sufficient to solve the clustering and labelling problems.

In Chapter~\ref{chap:eng-sp}, I explore potential cues from prosody and parents' behavior that might help infants identify questions. 


\section{Future direction}
\subsection{Jointly learn clause types and speech acts}
I hypothesis that children learn to identify speech act and clause type,
especially questions and interrogatives, in tandem and mutually informative ways: children
learn to identify clause types by tracking formal regularities in conjunction with their growing
knowledge of speech act and its associated socio-pragmatic cues; similarly, they learn
to identify speech acts by tracking socio-pragmatic cues in conjunction with their growing
understanding of the morpho-syntactic cues of various clause types.
