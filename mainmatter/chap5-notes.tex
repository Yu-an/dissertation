I focus on the distinction between questions and assertions in this chapter, and plan to come back to requests/commands in future work.

As reviewed in Chapter~\ref{chap:background}, children seem to be able to tell questions from other speech acts (\notdone{cite}) by using the morpho-syntactic cues of interrogative clauses. What leads children to this success? Before answering this question, we first need to establish what kinds of signals are available in the input to children. 

you are developing a theory in which the child has some prior ideas about how communication works. The child has an understanding/expectation that of the many differents kinds of activities that one can do with language, there are three acts that privileged (assertion, question, request) because of their relation to grammar (the clause types). These acts are associated with particular information states of speakers, which in turn means that you can expect different kinds of behavior to be correlated with these acts. Specifically, you can expect that when one asks a question that they are more likely to pause to await a response. Similarly, you can expect that when one asks a question they are more likely to look at the person they are seeking the information/response from because they are awaiting a response from that person. Thus, armed with a theory of what questions do, the child can expect certain kinds of nonlinguistic behavior to be somewhat correlated with the act of asking a question. This chapter examines two of those kinds of nonlinguistic behaviors to see whether they are indeed correlated with specific speech acts. To the extent that they are, then it is plausible that a child armed with a theory of speech acts could use those behaviors as indirect evidence about the acts themselves. And, in principle, this little bit of information about the acts could then be used to provide the 20% of pragmatic information that the child needs in order to get the clause type clusters identified accurately. That is, tie the whole thing together here. Each source of information you are studying is related to a single larger learning theory. Make sure you are making those connections clear.



The canonical use of questions is to seek information (\citealt{searle1975tax}, \citealt{levinson1983}, see \citealt{krifka2011q} for a recent overview), but when talking to children, especially pre-linguistic infants, parents can’t genuinely expect them to be in a position to provide that information (\citealt{holzman1972, shatz1978comprehension, tamir1980, yu2019pedagogical}, a.o.). %So how do children figure out the canonical mapping between the formal features of interrogatives and their conventionalized function as questions?  


Learners of both (perhaps all) languages face a further difficulty. Besides their main function of eliciting information, interrogative clauses can also be used to fulfill a range of other conventional functions. Test questions like (\ref{bg-prag:test}) do not present the asker as lacking information; parents can use questions like (\ref{bg-prag:ped}) to teach children object labels; indirect requests like (\ref{bg-prag:req}) are not genuinely soliciting information; and rhetorical questions act similar to assertions (\ref{bg-prag:rhe}). Such uses of interrogatives might mask the association of interrogatives with information-seeking speech acts. Additionally, other clause types like declaratives can be used as questions (e.g. rising declaratives, \citealt{gunlogson2004,gunlogson2008,jeong2018,rudin2018}). All these are potential sources of noise in the input with respect to the mapping between form and function.

\vspace{-2ex} 
\noindent
\begin{minipage}[t]{0.4\linewidth}	
\bex{bg-prag:types}
\bxl\label{bg-prag:test}
What is H_{2}O?		\hfill \tit{Test}
\ex \label{bg-prag:ped}
What's this?	\hfill \tit{Pedagogy}
\exl
\eex
\end{minipage}
\hspace{1cm}
\begin{minipage}[t]{0.45\linewidth}
\bex{}
\bxl\label{bg-prag:req}
Can you pass the salt?			\hfill \tit{Request}
\ex \label{bg-prag:rhe}
Are you crazy?	\hfill \tit{Rhetorical}
\exl
\eex
\end{minipage}
%\vspace{-1ex}

\noindent However, common to all the functions listed above is the fact that in a conversation, questions expect responses (\citealt{duncan1972turn}). In turn-taking theory, questions usually mark the turn-transition points: after a question, the current speaker yields their turn and appoint the next speaker while the other participants of the conversation would pick up the turn by answering the question (\citealt{kendon1967gaze, argyle1972gaze, levinson1983, tice2011turn} among others, see \citealt{enfield2010} for a survey).  Additionally, questions are argued to set up the topics and issues in a conversation (\citealt{roberts2012,farkasbruce2010}), and speakers could use questions to keep track of where they are in a conversation. Therefore, despite the fact that questions can be used to fulfill many functions, the role of questions in a conversation seems to be clear: speakers use questions to solicit responses and set up topics for discussion. If children are sensitive to signals of response expectations or topic settings, they could make use of that information to identify questions, and, in turn, interrogatives. 
%As mentioned above, combining syntactic and prosodic information may explain some mismatches, as with the case of rising declaratives. Other syntactic markers might be helpful as well: many have noted that interrogatives that function like imperatives tend to take specific forms such as having modals like (\ref{bg-prag:req}), or \tit{why don't} and \tit{why not} interrogatives (i.e. \tit{whimperatives}, \citealt{sadock1974, green1975whimp}).

In summary, children might face many challenges when learning interrogatives and their mapping to questions: English-speaking children should identify word order as a cue for interrogativity, but this information might be buried under many exceptions; Mandarin-speaking children should identify the markers for interrogativity (e.g.~\twh{}) despite some of these markers occurring in non-interrogative environments. Moreover, in both languages, children have to recognize the link between interrogativity and questionhood, which might be masked by exceptions in their input. 


As mentioned above, one challenge in the acquisition of questions is that the primary function of questions is to solicit information, but with pre-linguistic infants, parents cannot be expected to get answers to genuine information-seeking questions. 

Under our hypothesis, children learn to recognize questions via the expectation that questions are topic starters and require responses. The first property, questions as topic starters, would predict that if we examine the functions that each utterance performs, questions would tend to be used in situations where the speaker wants to direct the hearer's attention to a new object to start a new topic. 



%%%%%

For our \subhypos{} to work, it seems that we have to assume that in order to have figured out the clause type categories, children need to have figured out how to obtain speech act information. But this might lead us to a chicken-and-egg problem.

While some evidence suggests that 18-month-olds can infer speech acts, especially questions (\cite{casillas2017turn, marshmallowqueen}), their inferences might not be perfect, so they might misidentify some or many speech acts. That is, children might have only limited access to speech act information. Moreover, there is the problem of how children can infer speech act categories in the first place -- and it is undeniable that the clause type information is useful for solving this problem. As adults, the primary way we identify the speech act performed by a given sentence is through its clause type. But this is precisely the problem that the child is trying to solve (i.e. identifying the clause type). If children need speech act information to learn to identify clause type categories, but they also need clause type information to identify speech act categories, it seems that we have a chicken-and-egg problem. 

This way of putting the problem makes it seem like children have to either first train to recognize speech acts and, having done this, move on to learn to recognize clause types, or first train to recognize clause types and, having done this, move on to learn to recognize speech acts. But it is not a given that 
that learning clause types and learning speech acts happen sequentially like that. Conversely, it is likely that children learn to identify speech act and clause type in tandem and mutually informative ways: children learn to identify clause types by tracking formal regularities in conjunction with their growing
knowledge of speech act and its associated socio-pragmatic cues; similarly, they learn
to identify speech acts by tracking socio-pragmatic cues in conjunction with their growing
understanding of the formal features of various clause types.


To get one step closer from our \subhypos{} to this \tbf{\hypos{}}, I first ask how much speech act information children need to identify clause types. If children do not need \emph{perfect} speech act information to figure out clause types, then at least half of the \hypos{}, namely that children learn to identify clause types by tracking formal regularities in conjunction with their growing
knowledge of speech act, could be feasible. 
In Chapter~\ref{sec:engcl:model:noisy} and Chapter~\ref{sec:mancl:model:noisy}, I simulate the learning of clause type with various degrees of noise in the speech act information, so that we can see how much pragmatics a learner needs to succeed at the clustering problem. 

I then %address the second problem
ask what kind of non-clause type cues for speech act information is present in the input. Even if children must rely on clause type information to figure out the speech acts, they could have access to additional information that is unrelated to clause typing, but informative for recognizing speech act type. For example, in conversations, we use questions to elicit responses and information, which leads to behaviors like pausing after questions, or looking directly at our interlocutor, to nudge them to answer our questions. 
This in turn means that we can expect different kinds of behavior to be correlated with these acts. Thus, armed with a theory of what questions do (e.g. \cite{carruthers2018q}), the child can expect certain kinds of nonlinguistic behavior to be somewhat correlated with the act of asking a question. 

Some candidates for cues that could potentially differentiate questions from other speech acts are prosody, pauses, and direct eye gaze. Cross-linguistically, pitch rises tend to signal questions and pitch falls signal assertions; and some argue that this university reflects the innate knowledge that high pitch connects to the speech act of questioning (\cite{ohala1984,gussenhovenchen2000,gussenhoven2002} among others). If children are armed with the knowledge that questions tend to be associated with rising contours, they might expect rising contours to be somewhat correlated with the act of asking a question. 

The canonical function of questions is to solicit responses or seek information (\citealt{searle1975tax,levinson1983,stivers2010}, see \citealt{krifka2011q} for a recent overview). When we use questions, it is likely that we pause after questions, or look directly at our interlocutor, as a way to signal to them that they need to take over the conversational turn. If children have prior knowledge that certain speech act is used for response-elicitation, and if they know that pauses and direct eye gaze are how humans signal their expectations of a reciprocation of communication, then they may expect questions to be correlated to some degree with longer pauses and direct eye gaze.

If children have such expectations, would they find anything in the input? In Chapter~\ref{chap:eng-sp}, 
I conduct a corpus study examine the prosody of English-speaking parents' utterances (Section~\ref{sec:engsp:results:prosody}), length of pause after utterances (Section~\ref{sec:engsp:results:pause}), and proportion of eye gaze around the time of an utterance (Section~\ref{sec:engsp:results:gaze}). 


I find that parents do not use final rises more often with questions, but polar interrogatives have more final rises than other types of speech acts and clause types, including \twh-interrogatives and declaratives. Moreover, parents tend to pause longer after questions, and attend the child more when asking questions. To the extent that they are, it is in principle plausible that (a) a child could use these features, in addition to their growing knowledge of clause types to infer the speech act category of an utterance; and (b) in principle, this little bit of information about speech act could then be used to provide the 20\% of pragmatic information that the child needs in order to get the clause type clusters identified accurately. 

