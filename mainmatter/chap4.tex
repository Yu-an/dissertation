\chapter{Learning to identify clause types in Mandarin}
\label{chap:man-cl}

\section{Background}
\label{sec:mancl:bg}
\subsection{Clause types in Mandarin}
\label{sec:mancl:bg:theory}


While all languages have interrogative clauses, the formal features associated with this clause type vary. Mandarin interrogatives have different formal features than English interrogatives. For \twh-interrogatives, Mandarin \twh-phrases (\citealt{huang1982, cheng1991} among many others) do not need to be fronted to clause-initial position. Compare the declarative sentence in (\ref{bg-syn:dec-man}) with the \twh-interrogative in (\ref{bg-syn:wh-man}), the \twh-phrase \tit{shenme} occurs in the same position in the interrogative (\ref{bg-syn:wh-man}) as the noun phrase \tit{zaocan} in (\ref{bg-syn:dec-man}). 


%\begin{minipage}[t]{0.45\linewidth}	
\bex{bg-syn:dec-man}
\gll Xiaoxiao	chi-le zaocan.\\
Xiaoxiao eat-\Asp{} breakfast \\
``Xiaoxiao ate breakfast.''
\eex
\bex{bg-syn:wh-man}
\gll Xiaoxiao	chi-le \tun{shenme}.\\
Xiaoxiao eat-\Asp{} what \\
``What did Xiaoxiao eat?''
\eex


Polar interrogatives in Mandarin also have SVO word order, but are distinguished from declaratives by surface features like the question-forming particle ma or the A-not-A construction, as in (\ref{bg-syn:ma}) and (\ref{bg-syn:anota}).

\bex{bg-syn:ma}
\gll Xiaoxiao	chi-le	zaocan		\tun{ma}\\
Xiaoxiao	eat-ASP	breakfast	\Sfp{}\\
``Did Xiaoxiao eat breakfast?''
\eex
\bex{bg-syn:anota}
\gll Xiaoxiao	\tun{chi-mei-chi}	zaocan?\\
	Xiaoxiao	eat-\Neg-eat	breakfast\\
	``Did Xiaoxiao eat breakfast?''
\eex


Sentence final particles (SFPs) like \ma{} in Mandarin are optional particles at the right edge of a clause (\citealt{chao1968, zhudexi, huang1982, cheng1991, liboya2006} among others). Many SFPs can occur in any clause type, but \ma{} can only occur in polar interrogatives that do not have the A-not-A form seen in (\ref{bg-syn:anota}). Because of these differences, Mandarin-speaking and English-speaking children must pursue different strategies: English learners might mostly rely on word order information to distinguish interrogatives and declaratives, and children learning Mandarin might rely on interrogative phrases like \tit{ma}. 

However, Mandarin learners have to contend with an additional difficulty, namely some of these phrases are not uniquely associated with interrogativity. Two prominent examples are \twh-phrases, which can have interrogative and non-interrogative interpretations in Mandarin, and the question particle \ma, which has a homophone that can occur in non-interrogative clauses.


\bex{bg-syn:ambwh}
\gll Xiaoxiao	mei	chi	\tun{shenme}\\
Xiaoxiao	\Neg{}	eat	what\\
a.	What didn't Xiaoxiao eat?\\
b.	Xiaoxiao didn't eat anything.\\
\eex
\bex{bg-syn:ambma}
\gll Xiaoxiao	chi-le		zaocan		ma\\
	Xiaoxiao	eat-\Asp{}	breakfast	\Sfp{}\\
a.	Did Xiaoxiao eat breakfast?\\
b.	(Obviously) Xiaoxiao ate breakfast!\\
\eex 
The sentences in (\ref{bg-syn:ambwh}) and (\ref{bg-syn:ambma}) can be read as either questions or assertions. Thus, observing the \twh{} \tit{shenme} or sentence final \ma{} is not sufficient for a learner to determine the clause type. Learners must rely on additional cues from the context or prosody to determine the correct interpretation. In (\ref{bg-syn:ambwh}), when the context is compatible with both readings, assigning prosodic prominence (normally associated with focus) to the \twh-phrase is indicative of the interpretation as a question; in (\ref{bg-syn:ambma}), the question reading is associated with rising intonation, whereas the assertion reading is associated with falling intonation. 


Based on our discussion on Mandarin interrogatives in Section 2, we do not expect a word order difference between interrogatives and declaratives in Mandarin; instead, interrogativity is indicated by the use of markers such as sentence final particle \ma{}, A-not-A construction, \twh{}. Therefore, instead of coding the word order of the sentence, we will annotate whether or not the clause has an interrogative marker. As mentioned in Section~\ref{sec:bg}, besides \ma{}, SFPs are generally compatible with all clause types. For example, the A-not-A interrogative in (\ref{code-man:intanota}) contains the SFP \tit{a}, but this particle can be used in declaratives and imperatives as well (\citealt{chao1968, lithompson}). However, data from our pilot study suggests that SFPs are frequently found in interrogative clauses: among the 68 sentences with an SFP, 86\% are interrogatives like (\ref{code-man:intanota}). Although the particle \tit{a} is not by itself indicative of interrogativity, children may be able to exploit a correlation between sentence final particles and interrogativity to learn new question markers. That is, if they encounter an utterance like (\ref{code-man:intanota}) and have not yet learned A-not-A to be a polar interrogative indicator, the presence of the SFP \tit{a} could nudge them in the right direction. As of yet, this is an untested hypothesis, but if the pattern shown in the pilot is replicated in a bigger sample, further research will benefit from having this coding in the dataset produced by this project.


\bex{code-man:intanota}
\gll Ni	\tun{you}	watuji	\tun{mei-you}	a?	\\
You have	excavator	{\Neg{}	have}	\Sfp\\
``Do you have an excavator?'' \hfill \tsc{A-not-A}
\eex



In addition to the features mentioned above, we will also annotate a set of other morpho-syntactic features following \cite{perkins2019} as input into our model in Study 3. These are relevant properties of subjects, verbs and functional words that infants are known to represent prior to 18-months of age (Table~\ref{tb:syn}). Each feature is annotated as 1 if present, and 0 otherwise.
\begin{table}[H]
\caption{Syntactic features}
\vspace{-2.5ex}
\begin{center}
\begin{tabular}{c|p{7cm}|p{7cm}}
 \hline
  \hline
 & English & Mandarin \\
 \hline
Subject & Subject of known verb is overt in canonical subject position; sentence-initial; preceded by an auxiliary; preceded by another noun (e.g. topic NP) & Subject of known verb is overt in canonical subject position; preceded by another noun (e.g. topic NP) \\
\hline
Verb & Known verb is first verb in sentence; followed by direct object;  followed by a preposition or particle; has \tit{-ed, -en, -ing, -s}, or irregular morphology; verb is preceded by \tit{to, be, have, get}, or \tit{do}  & Known verb is first verb in sentence; followed by direct object; followed by an aspect marker or preceded by \tit{zai} \\
 \hline
Other & unknown function word sentence-initially, sentence medially before verb, sentence-medially after verb, or sentence-finally & unknown function word sentence medially before verb, sentence-medially after verb; unknown function word sentence-finally \\
 \hline \hline
\end{tabular}
\end{center}
\label{tb:syn}
\end{table}%


\subsection{Mandarin-acquiring children's knowledge}
\label{sec:mancl:bg:child}

\subsection{Roadmap of the chapter}
\label{sec:mancl:bg:map}

\section{Corpus study}
\label{sec:mancl:corpus}

\subsection{Methodology}
\label{sec:mancl:corpus:method}
\subsubsection{Clause Type}
\subsubsection{Morpho-syntactic cues for clause typing}

\subsection{Results}
\label{sec:mancl:corpus:results}
\subsubsection{Overview}
\label{sec:mancl:corpus:results:mapping}


\subsubsection{Morpho-syntactic cues}
\label{sec:mancl:corpus:results:syn}

\subsubsection{Informativeness of the cues}
\label{sec:mancl:corpus:results:supervised}


\section{Modeling the learning of clause type in 
Mandarin}
\label{sec:mancl:model}


\subsection{Performance of the \dlearnerabbr{} model}
\label{sec:mancl:model:results:d}

\subsection{Performance of the \plearnerabbr{} model}
\label{sec:mancl:model:results:d}

\subsection{Simulations with noisy pragmatic information}
\label{sec:mancl:model:results:noisy}

\section{Discussion}
\label{sec:mancl:discussion}




