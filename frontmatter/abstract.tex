% -*- mode: latex; coding: utf-8; fill-column: 72; -*-

Languages tend to have three major clause types (declaratives, interrogatives, imperatives), dedicated to three main speech acts (assertions, questions, commands). This mapping between  However, the particular forms that these clause types take differ from language to language, and have to be learned. Previous experimental results suggest that by 18 months old, children differentiate these clause types and associate them with their canonical speech act. This dissertation investigates how children learn to identify different clause types and speech acts. 

To learn clause types, children need to identify the right categories of clauses (the ``clustering problem") and figure out what speech act they are canonically used for (the ``labeling problem"). I investigate the extent to which learners need to rely on pragmatic information (i.e., knowing what speech act a given utterance of sentence is conveying), to solve not just labeling, but the clustering itself. I examine the role of pragmatics computationally by building two Bayesian clustering models. I find that for morpho-syntactic and prosodic information is not enough for identifying the right clause type clustering, and that pragmatics is necessary. I applied the same model to a morphological impoverished language, Mandarin, and found that the model without pragmatics performs even worse. Speech act information is crucial for finding the right categories for both languages. Additionally, I find that a little pragmatics goes a long way. I simulate the learning process with noisy speech act information, and find that even when speech act information is noisy, the model hones in on the right clause type categories, when the model without fails. 


But if speech act information is useful for clause type learning, how do children figure out speech act information? I explore what kind of non-clause type cues for speech act information are present in the input. Even if children must rely on clause type information to figure out speech acts, they could have access to additional information that is unrelated to clause typing, but informative for recognizing speech act type. When speakers perform speech acts, because of the conventional functions of these speech acts on the discourse, the performance might be associated with certain socio-pragmatic features. For example, because of questions' response-elicitation function, we might expect speakers to pause longer after questions. If children are equipped with some expectations about the functions of communication, and about what questions do, they might beable to use these socio-pragmatic cues to figure out speech act. 

I explore two cues that could potentially differentiate questions from other speech acts: pauses, and direct eye gaze. I find that parents tend to pause longer after questions, and attend to the child more when asking questions. Therefore it is in principle plausible that there are some socio-pragmatic features that children can use, in addition to their growing knowledge of clause types to infer the speech act category of an utterance. This little bit of information about speech act could then be used to provide the information that the child needs in order to get the clause type clusters identified accurately.
%This dissertation We find (a) that a learner must have access to some pragmatic information in order to find the right clause types but (b) this learner can succeed with very limited access to pragmatic information. 

% Local Variables:
% TeX-engine: xetex
% LaTeX-biblatex-use-Biber: t
% TeX-master: "../main"
% End:
