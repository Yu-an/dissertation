%As we will see in detail in the next chapter, infants as young as 18 months old seem to have figured it out. Given this early success, However, figuring out what the relevant clause types and speech acts are is not trivial. 


Clause types are grammatical categories that are thought of as features on $C^{0}$ (for example, as a [\textpm int] feature,  cf. \cite{katzpostal1964, baker1970int, langacker1974q, chomsky1995, rizzi1997, rizzi2001int, chomskylasnik1977, cheng1991}), what children need to figure out is which sentences have the [+int] value and which ones take the [-int] value. In English, the [+int] value of $C^{0}$ is associated with the syntactic rule of subject-aux inversion. So the child must come to treat (\ref{engcl:cluster}) and (\ref{engcl:cluster}a) alike (because of subject-aux inversion triggered by [+int] in $C^{0}$) and distinct from (\ref{engcl:cluster}b)

First, there is the clustering problem, namely learners need to be able to identify the abstract clause type category of a sentence. For example, while (\ref{ex:intro:cluster-base}) and (\ref{ex:intro:cluster}a) are two different strings, the learner needs to recognize that they share the same feature when it comes to clause types, frequently analyzed as a [+int] feature on $C^{0}$. They also need to recognize that even though (\ref{ex:intro:cluster}a) and (\ref{ex:intro:cluster}b) share the same lexical items, their $C^{0}$ are marked with different features: [+int] for the former, and [$-$int] for the latter.

%%%%%speech act stuff
- Children need to figure out the conventional effects that certain speech acts have on the current conversation
-- for example, questions require answers, assertions update your belief, requests/commands may add things to your to-do list.
- associate speech act categories with clause types
--- Children need to figure out what speech act is expressed by what kind of clauses with what kind of prosody

The canonical use of questions is to seek information (\citealt{searle1975tax}, \citealt{levinson1983}, see \citealt{krifka2011q} for a recent overview), but when talking to children, especially pre-linguistic infants, parents can’t genuinely expect them to be in a position to provide that information (\citealt{holzman1972, shatz1978comprehension, tamir1980, yu2019pedagogical}, a.o.).


%As our example above shows, the same sentence can be used to perform various speech acts depends on the speaker's intentions and the context under which the sentence is used. While we cannot take the speaker and the context out of the performance of a speech act, it is undeniable that the clause type of a sentence constrains the kinds of speech acts that can be performed.  


One way to think of this connection is to look at how utterances affect the context they are made in (\cite{hamblin1971, stalnaker1978, lewis1979scorekeeping, gazdar1981speech, roberts1996, portner2004, farkasbruce2010} a.o.). 

As \textcite{chierchia1990textbook} points out, while a sentence like (\ref{ex:intro:chierchia}) can be uttered to claim, guess, remind, warn, or threaten, there is a similarity across all these cases. Namely, that a declarative sentence S places the proposition expressed by S ``in the common ground and discards any possibilities rendered no longer live because of their inconsistency with that (possibly new) information'' (\cite[p.171]{chierchia1990textbook}). 

\begin{quote}
Although grammatical entities do not have semantic definitions in adult grammars, it is possible that such entities refer to identifiable semantic classes in parent-child discourse [\ldots] If the child tentatively assumes these syntax-semantics correspondences to hold, and if they do hold, he or she can make the correct inferences in the example. \textcite[p.39]{pinker1984}
\end{quote}

\bex{ex:intro:chierchia}
The bull is in the field.
\eex

Table~\ref{tab:portner2004} summarizes the clause types and their canonical functions, and their conventionalized effects on the discourse.

\begin{table}[H]
\begin{center}
\begin{tabular}{l|l|p{8cm}} 
\hline 
& Canonical Function & Conventionalized effect on the discourse \\
\hline
Declarative & Assertion & Proposing to add proposition to the common ground \\ 
\hline
Interrogative & Question & Add current question to the Question Under Discussion stack \\
\hline
Imperative & Request & Add (or propose to add) certain property to the To-Do List of the addressee \\ 
\hline
\end{tabular} 
\end{center}
\caption{Clause types and their conventionalized force; adapted from \textcite[p.238]{portner2004}}
\label{tab:intro:portner2004}
\end{table}%


Languages typically employ a wide range of clausal constructions -- actives, passives, imperatives, polar and \twh-interrogatives, clefts, pseudoclefts, relative clauses, and so on. Nonetheless, many of these distinctions are collapsed from the perspective of clause typing. For instance, verbs that select for declarative complements do not distinguish between actives, passives and clefts. Children, then, need to cluster the right morphosyntactic properties that are relevant for clause types and associate them with the right pragmatic function.




%%%%%%%%%%%%%%%%%%%%%%%%%
In the principle and parameter approach, a way to solve this cross-linguistic variation problem is to propose a parameter that covers all the variations. The learners are given choices for how to type a clause (i.e. assign [+int] or [-int] to $C^{0}$) by setting the value of a parameter. For example, \textcite{cheng1991} proposes the Clause Typing Hypothesis:

\begin{quote}
Every clause needs to be typed. In the case of typing a \twh-question, either a \twh-particle in $C^{0}$ is used or else fronting of a \twh-word to the Spec of $C^{0}$ is used, thereby typing a clause through $C^{0}$ by Spec-head agreement. \hfill \textcite[p.29]{cheng1991}

\end{quote}


According to this hypothesis, \twh-movement languages such as English assign a [$+$int] ([+wh] in Cheng's terminology, but it includes all interrogative clauses) value to the $C^{0}$ of sentences like (\ref{ex:intro:cheng-eng}) by moving the \twh-phrase to clause-initial position.\footnote{Note in other theories of [+int] and \twh-movement (e.g. \cite{chomsky1995}), \twh-movement does not assign [+int] to $C^{0}$, but is motivated by [+int]. } Meanwhile, \twh-in-situ languages achieve the same goal by utilizing a \twh-particle, \tit{ne}, as illustrated in (\ref{ex:intro:cheng-man}).

\bex{ex:intro:cheng-eng}
Who can Ann hug?
\eex
\bex{ex:intro:cheng-man}
\gll
Ann neng baobao shui (ne)\\
Ann can hug who \Sfp\\
\trans ``Who can Ann hug?''
\eex

%Putting aside the controversy about \twh-particles (cf. \cite{bruening2007wh, yangyang2018}), 
As \twh-particles are optional in many in situ languages, Cheng associates a particular learning strategy with this hypothesis: learners use the presence of polar question particle (e.g. \tit{ma} in Mandarin) to learn

since languages have two ways of realizing [+int], learners identify the abstract clause type feature by observing whether their language allows \twh-particles. 


In this dissertation, we explore a different way of solving the clustering problem, namely whether they can use their ability to track the distribution of certain features. Specifically, learners need to use pragmatic information (i.e. the speech act of the sentence), in conjunction with observations of distributions of morpho-syntactic features in the surface form of sentences, to infer clause type clusterings. We will return to this hypothesis later. 



%distinguish interrogatives and declaratives formally by as early as 12 months (\cite{geffenmintz2015wordorder}), and associate interrogatives with questions by age 3, and possibly as early as 18 months (\citealt{tyack1977, ervintripp1978, berningergarvey1981, rowland2003cdswh, seidl2003wh, casillas2013,casillas2017turn, clark2015turn, lammertink2015turn, gagliardi2016wh, perkins2020filler}, among many others). 

%Although there have been studies on parents' use of speech acts like questions, how children could use to acquire the distinction between declaratives and interrogatives and their association with their conventional functions remains under-explored. 
 
 %By providing data on how parents use questions in child-directed speech, and modeling how children use different signals in this input to learn questions and interrogatives, this project will deepen our understanding of the role of questions in language acquisition and beyond. 
%By systematically examining different aspects of language, from syntax, prosody to pragmatics, and modeling the learning process computationally, this study will provide insights into how children come to learn speech acts and clause types. Additionally, by comparing two languages in which the formal features of interrogativity differ greatly, this research will also advance our understanding of the universal and language-specific problems that children face when learning interrogatives. Finally, the labelled datasets as well as the the time-aligned corpora will be made available to the public for research on children's input beyond the learning of speech acts and clause types. 

%

