\chapter{Learning to identify clause types in Mandarin}
\label{chap:man-cl}

\section{Background}
\subsection{Clause types in Mandarin}


\subsection{Interrogatives in English and Mandarin}\label{bg:syn}
	
\noindent While all languages have interrogative clauses, the formal features associated with this clause type vary. 
\noindent Mandarin interrogatives have different formal features than English interrogatives. For \twh-interrogatives, Mandarin \twh-phrases (\citealt{huang1982, cheng1991} among many others) do not need to be fronted to clause-initial position. Compare the declarative sentence in (\ref{bg-syn:dec-man}) with the \twh-interrogative in (\ref{bg-syn:wh-man}), the \twh-phrase \tit{shenme} occurs in the same position in the interrogative (\ref{bg-syn:wh-man}) as the noun phrase \tit{zaocan} in (\ref{bg-syn:dec-man}). 


\vspace{-1.5ex} 
\noindent
\begin{minipage}[t]{0.45\linewidth}	
\bex{bg-syn:dec-man}
\gll Xiaoxiao	chi-le zaocan.\\
Xiaoxiao eat-\Asp{} breakfast \\
``Xiaoxiao ate breakfast.''
\eex
\end{minipage}
\hspace{0.5cm} %%这中间不能空行,否则不能compile
\begin{minipage}[t]{0.5\linewidth} 
\bex{bg-syn:wh-man}
\gll Xiaoxiao	chi-le \tun{shenme}.\\
Xiaoxiao eat-\Asp{} what \\
``What did Xiaoxiao eat?''
\eex
\end{minipage}
\vspace{0.5ex}

\noindent 
Polar interrogatives in Mandarin also have SVO word order, but are distinguished from declaratives by surface features like the question-forming particle ma or the A-not-A construction, as in (\ref{bg-syn:ma}) and (\ref{bg-syn:anota}).

\vspace{-2ex} 
\noindent
\begin{minipage}[t]{0.45\linewidth}	
\bex{bg-syn:ma}
\gll Xiaoxiao	chi-le	zaocan		\tun{ma}\\
Xiaoxiao	eat-ASP	breakfast	\Sfp{}\\
``Did Xiaoxiao eat breakfast?''
\eex
\end{minipage}
\hspace{0.5cm} %%这中间不能空行,否则不能compile
\begin{minipage}[t]{0.5\linewidth} 
\bex{bg-syn:anota}
\gll Xiaoxiao	\tun{chi-mei-chi}	zaocan?\\
	Xiaoxiao	eat-\Neg-eat	breakfast\\
	``Did Xiaoxiao eat breakfast?''
\eex
\end{minipage}
\vspace{1ex}

\noindent Sentence final particles (SFPs) like \ma{} in Mandarin are optional particles at the right edge of a clause (\citealt{chao1968, zhudexi, huang1982, cheng1991, liboya2006} among others). Many SFPs can occur in any clause type, but \ma{} can only occur in polar interrogatives that do not have the A-not-A form seen in (\ref{bg-syn:anota}). Because of these differences, Mandarin-speaking and English-speaking children must pursue different strategies: English learners might mostly rely on word order information to distinguish interrogatives and declaratives, and children learning Mandarin might rely on interrogative phrases like \tit{ma}. 

However, Mandarin learners have to contend with an additional difficulty, namely some of these phrases are not uniquely associated with interrogativity. Two prominent examples are \twh-phrases, which can have interrogative and non-interrogative interpretations in Mandarin, and the question particle \ma, which has a homophone that can occur in non-interrogative clauses.

\vspace{-2ex}
\noindent 
\begin{minipage}[t]{0.45\linewidth}	
\bex{bg-syn:ambwh}
\gll Xiaoxiao	mei	chi	\tun{shenme}\\
Xiaoxiao	\Neg{}	eat	what\\
a.	What didn't Xiaoxiao eat?\\
b.	Xiaoxiao didn't eat anything.\\
\eex
\end{minipage}
\hspace{0.5cm} %%这中间不能空行,否则不能compile
\begin{minipage}[t]{0.5\linewidth} 
\bex{bg-syn:ambma}
\gll Xiaoxiao	chi-le		zaocan		ma\\
	Xiaoxiao	eat-\Asp{}	breakfast	\Sfp{}\\
a.	Did Xiaoxiao eat breakfast?\\
b.	(Obviously) Xiaoxiao ate breakfast!\\
\eex
\end{minipage}
\vspace{-1.5ex}

\noindent 
The sentences in (\ref{bg-syn:ambwh}) and (\ref{bg-syn:ambma}) can be read as either questions or assertions. Thus, observing the \twh{} \tit{shenme} or sentence final \ma{} is not sufficient for a learner to determine the clause type. Learners must rely on additional cues from the context or prosody to determine the correct interpretation. In (\ref{bg-syn:ambwh}), when the context is compatible with both readings, assigning prosodic prominence (normally associated with focus) to the \twh-phrase is indicative of the interpretation as a question; in (\ref{bg-syn:ambma}), the question reading is associated with rising intonation, whereas the assertion reading is associated with falling intonation. 



\subsection{Mandarin-acquiring children's knowledge of clause types}

\subsection{Roadmap of the chapter}

\section{Corpus study}

\section{Models}
\subsection{The \distlearner{}}
\subsection{The \praglearner{}}
\subsection{Manipulating ratio of noise in pragmatic information}

\section{Discussion}





