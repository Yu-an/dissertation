\chapter{Introduction}
\label{chap:introduction}

Through language, humans are able to change the world by requesting that others do things, by asking for new information, and by providing new information themselves. It is hard to imagine what the human species would be without these fundamental linguistic capacities. Within all these speech acts that we use languages to perform, there seems to be three kinds of speech acts that are linguistically special. Indeed, languages tend to have dedicated three clause types for kinds of speech acts (\citealt{sz1985speechact, konig2007, aikhenvald2016, portner2018}, see \cite{konig2020} for a recent review) Specifically, declaratives are typically used for assertions (\ref{ex:intro:intro:dec}) and (\ref{ex:intro:man:dec}), interrogatives for questions (\ref{ex:intro:intro:int}) and (\ref{ex:man:intro:int}), and imperatives for commands (\ref{ex:intro:intro:imp}) and (\ref{ex:intro:man:imp}):

\bex{ex:intro:intro}
English clause types:
\bxl \label{ex:intro:intro:dec}
That's Elmo. \hfill Declarative, Assertion
\ex\label{ex:intro:intro:int} Is that Elmo? \hfill Interrogative, Question
\ex\label{ex:intro:intro:imp} Find Elmo! \hfill Imperative, Request
\exl
\eex

\bex{ex:intro:man}
Mandarin clause types:
\bxl \label{ex:intro:man:dec}
\gll Zhe shi Elmo.\\
This is Elmo\\
\trans ``This is Elmo." \hfill Declarative, Assertion
\ex \label{ex:intro:man:int}
\gll Zhe shi Elmo \tbf{ma}?\\
This is Elmo \Sfp\\
\trans ``Is that Elmo?'' \hfill Interrogative
\ex \label{ex:intro:man:imp}
\gll Zhizhi Elmo!\\
Point Elmo\\
\trans ``Point at Elmo!'' \hfill Imperative
\exl
\eex


As the above examples in English and Mandarin show, while both English and Mandarin have declaratives, interrogatives, and imperatives to be used to perform the functions of asserting, questioning, and commanding, the form of each clause types varies. For example, if we compare the interrogative clauses in the two languages with the declarative clauses, we can see that the English interrogative clause has a different word order from the declarative one, with the subject and the auxiliary switch places i the interrogative clause. In Mandarin, the difference arises at the edge of the sentence, as the interrogative clause has an additional sentence final particle \tit{ma} at the end of the sentence. Mandarin could also use A-not-A constructions for polar interrogatives:

\bex{ex:intro:anota}
\gll Zhe \tbf{shi-bu-shi} Elmo?\\
This be-\Neg-be Elmo\\
``Is this Elmo?'' \hfill A-not-A Interrogative
\eex

Again in this case, the canonical function of the sentence is questioning, same as its English counterpart in (\ref{ex:intro:intro:int}), but interrogativity in this example is marked by the presence of the disjunctive negative structures (i.e. \tit{shi-bu-shi}). 

Besides using relative word order of constituents (English), particles (Mandarin \tit{ma}), disjunctive negative structures (Mandarin A-not-A), we can also find languages like Italian and Portuguese where declaratives and interrogatives share the same morpho-syntactic features and only differ in intonation, and languages like West Greenlandic that differentiate the two clause types by verb inflection (\tit{vutit} for declaratives and \tit{vit} for interrogatives in (\ref{ex:intro:gl})):
\bex{ex:intro:gl}
West Greenlandic
\bxl
\label{ex:intro:gl:dec}
\gll neri- vutit\\
eat- \Ind.\Ssg.\Pst{}\\
``You ate.'' \hfill Declarative
\ex \label{ex:intro:gl:int}
\gll neri- vit\\
eat- \Int.\Ssg.\Pst{}\\
``Did you eat?'' \hfill Interrogative
\exl
\hspace*{\fill} \cite[18, ex (50)]{konig2007}
\eex

For another type of interrogative, \twh-interrogatives, languages differ in whether the \twh-phrase is fronted to the beginning of a clause,  like English (\ref{ex:intro:engwh}), or stays in situ, like Mandarin (\ref{ex:intro:manwh}):

\bex{ex:intro:engwh}
\tbf{What} did Elmo eat?
\eex
\bex{ex:intro:manwh}
\gll Elmo chi-le \tbf{shenme}?\\
Elmo eat-\Asp{} what\\
``What did Elmo eat?''
\eex

Despite all these cross-linguistic differences in how clause types are marked, we can still see that  in each language, the three major clause types (\diis{}) are canonically associated with the same three major speech acts (\aqrs{}). 

Of course, while each of these three clause types has their canonical mapping to a speech act, they can be used to perform other speech acts too. For example, when we use the sentence \tit{Can you pass the salt?} at a dinner table, even though the sentence is an interrogative in English with subject-auxiliary inversion, the speaker uses it to perform a requesting act. The same is true in Mandarin:

\bex{ex:intro:manmis}
\gll Keyi bangwo di-yixia zhijin \tbf{ma}?\\
Can help me pass-\Clf{} napkin \Sfp{}\\
``Can you pass me the napkin?''
\eex

Sentences with particle \tit{ma}, as discussed above, are interrogatives in Mandarin. But same as its English counterpart, when you utter (\ref{ex:intro:manmis}) to me when I'm sitting next to the napkin box, you are making a request rather than asking a question. 

Thus, languages tend to have dedicated clause types that are typically associated with dedicated speech acts, but these clause types can be used for other speech acts. 

As adults, we understand this canonical mapping between clause types and speech acts. When we hear someone utters ``Is it raining?'' we assume that they are asking a question, and when we hear someone says ``It's raining!'' we assume that they are making an assertion.  But for a child whose grammar is still developing, this might not be trivial, as they have to figure out the makeup of the \diis{} in their language and associate them to their canonical speech acts.   %How do they figure out speech act, if they can’t rely on clause types, and conversely, how can they figure out clause types, if they need clause types to figure out speech acts? how does a child figure out the clause types in their language, and link them to their canonical functions?

Remarkably, children seem to have figured this out early. By 18months, they seem to be able differentiate interrogatives from declaratives, and understand that people use interrogatives to ask questions, all while their grammar and understanding of the world are still in development (\cite{geffenmintz2011,geffenmintz2015wordorder,casillas2017turn,perkins2019,marshmallowqueen}. This dissertation therefore examines how children figure out clause typing. Specifically, children need to identify the right categories of clauses (the ``clustering problem") and figure out what speech act they are canonically used for (the ``labeling problem"). 

In the remainder of this chapter, I first explain these two learning problems related to clause type categories (Section~\ref{sec:intro:cl:problem}). As the mapping between clause types and speech acts seem to be robust across languages, I explore the necessity and feasibility of using the speech act information to figure out clause typing (Section~\ref{sec:intro:cl:prag}). In Section~\ref{sec:intro:hypo}, I detail my plan for probing the question of how children might figure out the right clause types, and address the remaining issue of learning speech acts in Section~\ref{sec:intro:sp}. Section~\ref{sec:intro:roadmap} summaries the discussion and lays out the roadmap of this dissertation. 

\section{Learning problems}
\label{sec:intro:cl}
\subsection{The clustering and labeling problem}
\label{sec:intro:cl:problem}

As we have seen, languages tend to have three clause types.  Given the quasi-universality of these main clause types, it may be reasonable to assume some kind of language universal, such that children would expect that their language is likely to have three main clause types for \aqrs{}. But even if we assume that the clause type categories are innate, learners still face two main problems.

First, languages do not wear these abstract clause type categories on their sleeves. So the learner need to identify the specific signals in the surface form of the sentences in their language associated with the three abstract categories. That is, they need to identify the right categories of clauses. I will refer to this problem as the \tbf{clustering problem}.
To use English as an example, English-acquiring children have to figure out that  (\ref{ex:intro:cluster-base}) is related to (\ref{ex:intro:cluster:dec}) even though they have different lexical items, because the subjects in both follow the auxiliaries; but also (\ref{ex:intro:cluster-base}) is different from  (\ref{ex:intro:cluster:dec}) even though both share the same lexical item, because the subject in the latter precedes the auxiliary. 

\bex{ex:intro:cluster-base}
Do you want a cookie?
\eex
\bex{ex:intro:cluster}
\bxl\label{ex:intro:cluster:int}
Is that Elmo?
\ex\label{ex:intro:cluster:dec}
That’s Elmo!
\exl
\eex

\begin{comment}
Meanwhile, Mandarin-acquiring children have to figure out that \ref{ex:intro:man:cluster-base}

\bex{ex:intro:man:cluster}
\eex
\bex{ex:intro:man:cluster-base}
\bxl
\gll Zhe shi Elmo.\\
This is Elmo\\
\trans ``This is Elmo." \hfill Declarative
\ex 
\gll Zhe shi Elmo \tbf{ma}?\\
This is Elmo \Sfp\\
\trans ``Is that Elmo?'' \hfill Interrogative
\exl
\eex
\end{comment}


Second, after identifying the clusters, learners need to determine the canonical function of each cluster in the system. That is, after clustering sentences into three categories, children still need to learn which one of these clusters is the interrogatives, which is the declaratives, and which is the imperatives. We will refer to this as the \tbf{labeling problem}. %As adults, when we hear someone say ``Is it raining?'' we assume that they are asking a question, and when we hear someone says ``It's raining!'' we assume that they are making an assertion.  But for a child whose grammar is still developing, this might not be trivial, as they have to figure out the makeup of the \diis{} in their language and associate them to their canonical speech acts.  


Solving these two problem aren't straightforward. For the clustering problem, as clause type categories do not have a one-to-one mapping with the morpho-syntactic properties in the surface form of sentences. For example, as we have discussed, in English, the hallmark of interrogativity is subject-auxiliary inversion, which can be seen in polar interrogatives (\ref{ex:intro:intro:int}) in the last section and \twh-interrogatives (\ref{ex:intro:engwh2}) below. However, this association of word order and interrogativity has many exceptions. Subject \twh-interrogatives like (\ref{ex:intro:engwhsubj}) do not have this formal feature:


\bex{ex:intro:engwh2}
Who \tit{can} \tbf{Sue} hug? \hfill Object \twh{}, Subject-Auxiliary Inversion
\eex
\bex{ex:intro:engwhsubj}
\tbf{Who} \tit{can} hug Ann? \hfill Subject \twh{}, no Subject-Auxiliary Inversion
\eex


Similarly, in embedded clauses, we also do not see subject-auxiliary inversion: 
\bex{ex:intro:eng-embed}
Mary wonders $[_{\Int}$ whether \tbf{Ann} \tit{can} hug Elmo].
\eex

As shown in (\ref{ex:intro:eng-embed}), the auxiliary \tit{can} and subject \tit{Ann} of the embedded interrogative have the same word order as a matrix declarative sentence.



Conversely, some morpho-syntactic properties typically associated with interrogative clauses could also appear in other settings. For example, in English, some declaratives exhibit subject-auxiliary inversion: 

\bex{bg-syn:dec}
\bxl\label{bg-syn:decreg}
Mary would never eat tripe in her life. 
\ex\label{bg-syn:decneg} Never in her life would Mary eat tripe.
\exl
\eex

Both (\ref{bg-syn:decreg}) and (\ref{bg-syn:decneg}) are declaratives, but when the negator \tit{never} is fronted as in (\ref{bg-syn:decneg}), the auxiliary precedes the subject.



Therefore, learners need to infer the right clause type category of sentences they hear in the input, but they might not see the crucial surface morpho-syntactic features for clause typing, or the surface features that they do see misalign with the actual clause type category of the sentence. 

But this many-to-many mapping problem is not the only challenge for the learner to solve the clustering problem. Learners also have to deal with cases where the relevant morpho-syntactic features are masked, for instance, by other syntactic operations. For example, left-edge-ellipsis is such an operation (\cite{zwickypullum1983leftedge}):

\bex{ex:intro:lee}
Want to go out
\eex

The string in (\ref{ex:intro:lee}) could be a result from eliding the subject pronoun from a declarative clause like (\ref{ex:intro:lee-unpack}a), or a result from eliding the subject and the auxiliary from a polar interrogative like (\ref{ex:intro:lee-unpack}b). But the surface form of (\ref{ex:intro:lee}) itself does not have enough information to help us identify its clause type feature.\footnote{Note that the intonation would not help in this case either, because both (\ref{ex:intro:lee-unpack}a) and (\ref{ex:intro:lee-unpack}b) are likely to have a final rising contour L* H-H\% (\cite{gunlogson2008, jeong2018, goodhue2021rd}). } 

\bex{ex:intro:lee-unpack}
\bxl
You want to go out?
\ex Do you want to go out?
\exl
\eex

%These problems are not specific to English. In fact, Mandarin provides lots of cases where the formal features for clause typing are misaligned or absent. %This problem is especially prominent with \twh-interrogatives/

\begin{comment}
In Mandarin, the presence of \twh-phrases and question particles are the two hallmarks of Mandarin interrogatives. But in \twh-interrogatives, the question particle \tit{ne} is optional, and \twh-phrases do not move to sentence initial position. As a result, in some cases like (\ref{ex:intro:man-ne}b), the only difference between the interrogative sentence and its declarative counterpart is the presence of \twh-phrase:

\bex{ex:intro:man-ne}
\bxl
\gll Elmo chi-le dian \tbf{shenme} \tbf{ne}?\\
Elmo eat-\Asp{} \Cl{} what \Sfp{}\\
``What did Elmo eat?''
\ex \gll Elmo chi-le dian \tbf{shenme}?\\
Elmo eat-\Asp{} \Cl{} what\\
``What did Elmo eat?''
\exl
\eex 

\bex{ex:intro:man-np}
\gll Elmo chi-le dian binggan\\
Elmo eat-\Asp{} \Cl{} cookie\\
``Elmo ate some cookies.''
\eex

However, declarative sentences could also have \twh-phrases, where these phrases are interpreted as indefinites like English \tit{any/a} (\cite{huang1982, cheng1991,LMYY2021}). %As a result, a string like (\ref{ex:intro:m-whamb}) could be either an interrogative (interpretation a) or a declarative (interpretation b). 


\bex{ex:intro:m-whamb}
\gll Xiaoxiao mei 	chi 	\tun{shenme} dongxi\\ 
Xiaoxiao \Neg{} 	eat	what	things\\
a.	``What didn’t Xiaoxiao eat?''	\hfill Interrogative \twh\\
b.	``Xiaoxiao didn’t eat anything.''		\hfill Indefinite \twh
\eex

In (\ref{ex:intro:m-whamb}), the interrogative version of the string 
\end{comment}

Thus, even if we assume that the learners come with the expectation that their language is likely to have three clause types, they still have to figure out the right clustering of the clauses (the clustering problem), and label the clusters with the canonical functions of the clauses (the labeling problem). They might face many challenges when solving the two problems, as the formal features for one clause type category might not show up in the surface string, or might appear in sentences of another category. Would children learn clause typing from surface features alone? 

As we have discussed, cross-linguistically, we see this mapping between the three major clause types and the three major speech acts. Might this source of information be helpful?

\subsection{Speech act might be helpful }
\label{sec:intro:cl:prag}

At the beginning of this chapter, we saw that cross-linguistically, declaratives are canonically mapped to assertions, interrogatives to questions, and imperatives to commands. Exploiting this mapping could potentially help the learner. Let's see how pragmatics might help for the solving the clustering and labeling problem respectively.

Clearly, speech act information will be necessary for the labeling problem: to understand that a particular clause type is declarative and another is an interrogative, it will be useful to know that the former is typically used for assertions, and the latter for questions. 

As for the clustering problem, surface formal features alone may allow a learner to cluster sentences into three distinct formal categories. However, as we discussed in the last section, the formal features for one clause type category might not show up in the surface string, or might appear in sentences of another category. The question then is, is the surface formal features of the sentences that the learners observe in their input sufficient for identifying the right three clause types? If not, what information might bridge this gap between learners' input and the abstract clause type categories they need to acquire, i.e. what information might help learners bootstrap into clause type categories (cf. \cite{pinker1984, gleitman1990, hacquardlidz2018})? 

The obvious candidate here seems to be speech act information. We have seen that the mapping between the three major clause types and the three major speech acts, so learners could take advantage of this mapping to fill in the gaps left by surface formal features in the input. 

\subsection{Speech act might be hurtful}
But this mapping between clause types and speech acts could also hurt children's chances of learning clause types, as this mapping is not inviolable. In some contexts, it is possible that the conventionalized speech act associated with a clause type is not the actual speech act performed by uttering it. \tbf{Indirect speech acts} are these mismatching cases where the primary, ``non-literal'' speech act of an utterance is different from the conventionalized, ``literal'' speech act of a sentence associated with its clause type (\citealt{searle1975tax, searle1976class, bachharnish1979, searlevanderveken1985, portner2004, starr2014, portner2018, murraystarr2020} a.o.). As we have discussed briefly at the beginning of the chapter,  when you utter \tit{Can you pass the salt?} at the dinner table, it is likely that you intent this utterance to be taken as a request. As an interrogative clause with subject-auxiliary inversion, its conventionalized (and ``literal'') act is questioning, but the primary act performed is requesting. As a result, some speech act categories can be expressed by more than one clause types, and vice versa. For example, interrogatives can express assertions, questions, requests/commands (\ref{eng-cl:int-all}); and questions can be expressed by \diis{} (\ref{eng-cl:q-all}).

\bex{eng-cl:int-all}
Speech acts expressed by interrogatives 
\bxl Is it snowing? \hfill Question
\ex Aren't you sweet. \hfill Assertion
\ex Can you pass the salt? \hfill Request
\exl
\eex
\bex{eng-cl:q-all}
Clause types expressing questions
\bxl
Is it snowing? \hfill Interrogative
\ex It's snowing? \hfill Declarative
\ex Tell me if it's snowing! \hfill Imperative
\exl
\eex

If such mismatching cases are prevalent in children's input, the speech act information might not be helpful for children to figure out clause types. 


\section{Learning clause types}
\label{sec:intro:hypo}

To summarize our discussion so far, we have seen that languages tend to have three clause types dedicated to three speech acts, and by 18 months old, children seem to be able to differentiate these clause types and associate them with their canonical speech act. To gain this ability, they need to identify the right categories of clauses (the "clustering problem") and figure out what speech act they are canonically used for (the "labeling problem"). To solve the labeling problem, the speech act information might be helpful, but if there are too many mismatching cases between speech acts and clause types, this information might not be helpful. To solve the clustering problem, children need to pay attention to the surface morpho-syntactic features of each sentence in their input. But in the input, the surface features might be absent or misleading. 

This dissertation therefore sets out to address these questions: first, are the surface formal features of the sentences in the input sufficient for children to figure out the clustering of clause types? Second, if not, is speech act information helpful or hurtful? 

I answer these question by comparing two learners, a \distlearner{} (\dlearnerabbr{}), and a \praglearner{} (\plearnerabbr{}). Both learners use the surface morpho-syntactic features of the input sentences to learn the clause type categories, but the \plearnerabbr{} additionally has access to the speech act that the sentence is used to perform. In Chapter~\ref{chap:eng-cl} I simulated these two learners with two Bayesian clustering models. By comparing the performance of the two models, I show that morpho-syntatic features alone is not sufficient, the speech act information is crucial (and helpful). 

In Chapter~\ref{chap:man-cl}, I test the models' performance with Mandarin, which has a different set of surface formal features for clause typing. Due to its impoverished morphological system, the surface formal features for clause typing might be even more likely to be absent or misaligned. The results from the two models suggest that the surface formal features are even less informative for clause typing; without speech act information, the learner might not be able to identify any clause types.  

This insufficiency of syntax leads us to the following \tbf{\subhypos{}}:
%\begin{comment}
\begin{quote}
Infants use the speech act information, in addition to observations of morpho-syntactic features in the surface form of sentences, to cluster and label input sentences into the three major clause types.
\end{quote}
%\end{comments}


This hypothesis states that to compensate for the insufficiency of surface formal features, children need to use the speech act information to bootstrap into clause type categories.



\section{A chicken-and-egg problem: Learning speech act categories}
\label{sec:intro:sp}
For our \subhypos{} to work, it seems that we have to assume that by the time children have figured out the clause type categories, they need to have figured out the speech act information. But this might lead us to a chicken-and-egg problem.

While some evidence suggests that 18-month-olds can infer speech acts, especially questions (\cite{casillas2017turn, marshmallowqueen}), their inferences might not be perfect. Consequently, they might have only limited access to speech act information. Moreover, there's the problem of how children can infer speech act categories in the first place -- and it is undeniable that the clause type information is useful for solving this problem. As adults, the primary way we identify the act performed by a given sentence is through its clause type. But this is precisely the problem that the child is trying to solve (i.e. identifying the clause type). If children need speech act information to identify clause type categories, but they also need clause type information to identify speech act categories, it seems that we have a chicken-and-egg problem. 

But it does not have to be that the learning of clause type and the learning of speech acts happen sequentially. It is likely that children learn to identify speech act and clause type in tandem and mutually informative ways: children learn to identify clause types by tracking formal regularities in conjunction with their growing
knowledge of speech act and its associated socio-pragmatic cues; similarly, they learn
to identify speech acts by tracking socio-pragmatic cues in conjunction with their growing
understanding of the formal features of various clause types.

To get one step closer from our \subhypos{} to this \tbf{\hypos{}}, I first ask how much speech act information children need to identify clause types. If children do not need \emph{perfect} speech act information to figure out clause types, then at least half of the \hypos{}, namely children learn to identify clause types by tracking formal regularities in conjunction with their growing
knowledge of speech act, could be true. 
In Chapter~\ref{chap:eng-cl} and Chapter~\ref{chap:man-cl}, I simulate the learning of clause type with various degrees of noise in the speech act information, so that we can see how much pragmatics a learner needs to succeed at the clustering problem. 

I then %address the second problem
ask whether what kind of non-clause type cues for speech act information is present in the input. Even if children must rely on clause type information to figure out the speech acts, but it could be that in addition to this information, there might be other cues that could help. In Chapter~\ref{chap:eng-sp}, I explore three of such cues: prosody (Section~\ref{sec:engsp:results:prosody}), pauses (Section~\ref{sec:engsp:results:pause}), and direct eye gaze (Section~\ref{sec:engsp:results:gaze}). I find that parents do not use final rises more often with questions, but polar interrogatives have more final rises than other types of speech acts and clause types, including \twh-interrogatives and declaratives. Parents tend to pause longer after questions, and attend the child more when asking questions.  Even though these cues cannot perfectly predict the speech act of an utterance, since my simulations show that even very noisy speech act information is useful for learning clause types, this result suggests a way out of the circularity. 


\section{Discussion and roadmap}
\label{sec:intro:roadmap} 
To summarize, this dissertation how English- and Mandarin-acquiring children figure out the make-up of the three major clause types in their language and link them to their canonical speech act. Languages tend to have three major clause types (declaratives, interrogatives, imperatives), dedicated to three main speech acts (assertions, questions, commands, \cite{sz1985speechact} among others). However, the particular forms that these clause types take differ from language to language, and have to be learned. Previous experimental results suggest that by 18 months old, children differentiate these clause types and associate them with their canonical speech act (\cite{geffenmintz2011,geffenmintz2015wordorder,casillas2017turn,perkins2019,marshmallowqueen}). To gain this ability, children need to identify the right categories of clauses (the \tbf{clustering problem}) and figure out what speech act they are canonically used for (the \tbf{labeling problem}). 

This dissertation investigates whether the surface formal features are sufficient for learning the right clause types, and if now, how much the learners need to rely on the speech act information. I address these questions computationally by building two Bayesian clustering models simulating the learning processes of English- and Mandarin-acquiring children. I find that morpho-syntactic information is not sufficient for acquiring clause type categories. A learner, especially a Mandarin learner, must have access to some pragmatic information in order to find the right clause types. I also show that even if the learner cannot perceive speech act information all the time, they can still benefit from this information (but with various degrees). I also demonstrate that prosody, length of pauses between utterances, and direct eye gaze could to some extent help children identify the speech act information.


%This project will focus in particular on interrogatives and questions, and how children distinguish them from declaratives and assertions. One reason for this is because the use of questions by parents of preverbal children raises interesting challenges. The canonical role of questions is usually assumed to be information-seeking (Searle 1969). However, before children can talk, parents can’t expect informative responses, and their questions are often ones that they know the answers to (Holtzman 1972, Shatz 1979, Tamir 1980). Another reason is that the two main types of interrogatives (polar vs. WH-) raise interesting questions about form: in languages across the world, including English, it is not entirely obvious that the two types are unified from a formal standpoint, as the formal features of one (e.g. rising prosody or question particle) do not necessarily carry over to the other, and furthermore, each shares formal features with declaratives that the other does not. For example, polar interrogatives in many languages are not syntactically distinguishable from declaratives, while WH-interrogatives in many languages frequently bear the same final falling intonational contour as declaratives (Bartels 1999, Hedberg et al. 2010, Truckenbrodt 2012). A third reason is that the quasi- universality of prosodic rises in polar interrogatives (Gussenhoven 2004, a.o.) makes it a good candidate for a universal that learners may be equipped with. If so, children may be able to identify polar interrogatives earlier than WH-interrogatives. This prosodic head start for polar interrogatives may be further aided by the fact that parents may use them more than WH- interrogatives for genuine information seeking (Walker & Armstrong 1994).

This dissertation is organized as follows. Chapter~\ref{chap:background} examines the developmental trajectory of speech acts and clause types, especially questions and interrogatives. As we will see, English-acquiring infants as early as 18 months seem to have already sensitive to the distinctions between different clause types and speech acts, and seem to understand the mapping between questions and interrogatives. The same holds for infants acquiring other languages as well, even though we have less evidence. Our question then is, how do 18-month-olds learn to figure out clause types?

Chapter~\ref{chap:eng-cl} looks at how English-acquiring 18-month-olds could solve the problem. Specifically, is information from syntax enough for children to find the right three clause type categories, or do they need pragmatic information like the speech act of the sentence to find the right clustering? I build two computational models to address this question, a \distlearner{} (\dlearnerabbr{}), and a \praglearner{} (\plearnerabbr{}). These two learners both need to infer the abstract clause type, but \dlearnerabbr{} draws inferences from syntactic information alone while \plearnerabbr{} uses both syntactic and pragmatic information. I use a corpus study to first provide a quantitative description of the type of input that infants get, and use the resulted annotated dataset as input for the computational models. I find that pragmatic information is indeed important for solving the clustering problem: without the speech act information, \dlearnerabbr{} cannot find the right clause types. Additionally, a little pragmatics goes a long way, as  even if 80\% of the pragmatic information is noise, it still improves the learner's performance. 

In Chapter~\ref{chap:man-cl}, I apply the same methodology to another language, Mandarin. Mandarin-acquiring infants figure out the clause types of their language around the same age as English-acquiring infants, but the two languages employ different morpho-syntactic features for clause typing. How do Mandarin-acquiring infants solve the problem? Do they also need pragmatic information? I compare the same two learners, and found that learners might not be able to identify any of the clause types correctly without pragmatic information; even with pragmatic information, the learner might still have some difficulty identifying the imperative clause type.

But in both chapters, I assume that infants have information about speech act types at their disposal. How do they obtain such information about the speech acts of their parents' utterances? In Chapter~\ref{chap:eng-sp}, I explore potential cues from prosody and parents' behavior that might help English-acquiring infants identify questions. I do not find the prosody of parents' questions differ from other types of speech acts, but there is a distinction between polar interrogatives and declaratives. The other two cues, length of pauses between utterances and direct eye gaze, both could distinguish assertions from questions/requests, but could not distinguish between questions and requests. Even though these three cues cannot perfectly predict the use of speech act, but as my simulation with noisy speech act information suggests, learners might still benefit from this source of information when learning the clustering of clause types.  Chapter~\ref{chap:discussion} concludes the dissertation.


